\documentclass[a4paper,12pt]{article}
\usepackage[slovene]{babel}
\usepackage[utf8]{inputenc}
\usepackage[T1]{fontenc}
\usepackage{amsmath}
\usepackage{amsfonts}
\usepackage{lmodern}
\usepackage{amsthm}
\usepackage{nopageno}
\usepackage{sectsty}
\usepackage[DIV=14,BCOR=2mm]{typearea}
\usepackage[normalem]{ulem}
\usepackage{marvosym}
\setlength\parindent{0pt}
\allsectionsfont{\centering}

\newtheorem*{trditev}{\textsc{Trditev}}
\newtheorem*{izrek}{\textsc{Izrek}}
\newtheorem*{posledica}{\textsc{Posledica}}
\newtheorem*{posl}{Posledica}
\newtheorem*{definicija}{\textsc{Definicija}}
\newtheorem*{dokaz}{\textsc{Dokaz}}
\pagestyle{plain}


\begin{document}

\begin{titlepage}
	\centering
	{\scshape\LARGE Univerza v Ljubljani \\}
	\vspace{10mm}
	{\scshape \normalsize Fakulteta za matematiko in fiziko\par}
	\vspace{40mm}
	{\scshape\large Teorija iz predavanj predmeta\par}
	\vspace{5mm}
	{\Huge\bfseries \textsc{Algebra 1} \par}
	\vspace{30mm}
	{\large\itshape Maja Levak\par}
	\vfill
	Predavatelj\par
	Doc.~dr.~\textsc{Klemen Šivic}

	\vfill

% Bottom of the page
	{\normalsize{Študijsko leto 2018/2019} \par}
\end{titlepage}


\newpage
\tableofcontents
\newpage

\begin{center}
\section{Vektorski prostor $\mathbb{R}^3$}
\subsection{Koordinatni sistem}
\end{center}

Model na množico $\mathbb{R}$ je številska premica ali realna os. \\

Premica na danem koordinatnem sistemu: na premici izberemo točko 0 (izhodišče) in 1 (enota). Običajno je 1 desno od 0. Vsaki točki $T$ na številski premici lahko priredimo neko realno število $x \in \mathbb{R} $. Če je $T$ desno od 0, točki priredimo razdaljo te točke od izhodišča. Če je $T$ levo od 0, priredimo nasprotno vrednost razdalje te točke od izhodišča. Izhodišču priredimo 0. Ta preslikava je bijekcija iz številske premice v množico $\mathbb{R}$, zato številsko premico identificiramo z realnimi števili. \\
$$  \mathbb{R}^2 =  \mathbb{R} \times \mathbb{R} = \{ (x,y) ; x, y \in \mathbb{R} \}$$  

\begin{center}
= množica urejenih parov realnih števil. \\
\end{center}

Model na $\mathbb{R}^2$ je ravnina z danim koordinatnim sistemom. Koordinatni sistem določata dve pravokotni številski premici, tako, da se sekata v izhodiščih obeh premic. Obe številski premici poimenujemo \textbf{koordinatni osi}. Običajno si enici izberemo desno od izhodišča in nad njim. Vodoravni osi rečemo \textbf{abscisna} ali \textbf{$x$-os}, navpični pa \textbf{ordinatna os} ali \textbf{$y$-os}. Tak koordinatni sistem imenujemo \textbf{pozitivno orientiran}. Če pozitivni poltrak $x$-osi zavrtimo za pozitivni kot 90 stopinj (v nasprotni smeri urinega kazalca), dobimo pozitivni poltrak. \\

Imenujemo poljubno točko v ravnini. Premica skozi $T$, vzporedna $y$-osi, seka $x$-os v natanko eni točki, ki ustreza natančno določenimu realnemu številu $x$. Podobno vodoravna premica. Skozi $T$ seka $y$-os v natančno eni točki, ki ustreza natančno določenemu številu $y$. Točki $T$ smo priredili urejen par $(x,y)$. Preslikava, ki točki $T$ priredi urejen par $(x,y)$ je \textbf{bijekcija iz ravnine v $\mathbb{R}^2$}. Zato ravnino identificiramo z $\mathbb{R}^2$. Številoma $x$ in $y$ pravimo koordinati točke $T$ in šišemo $T(x,y)$. \\

\textbf{Razdalja} med točkama $T_1(x_1,y_1)$ in $T_2(x_2,y_2)$: $$d(T_1,T_2) = \sqrt[2]{(x_2 - x_1)^2 + (y_2 - y_1)^2}$$ 

\textbf{Množica vseh urejenih trojic realnih števil}: $$\mathbb{R}^3 = \mathbb{R} \times \mathbb{R} \times \mathbb{R} = \{(x,y,z) ; x,y,z \in \mathbb{R} \}$$

Model za $\mathbb{R}^3$ je prostor z danim koordinatnim sistmom. Sestavljajo ga tri pravokotne številske premice, ki se sekajo v eni točki, ki je izhodišče vseh treh številskih premic. Temu presečišču pravimo \textbf{koordinatno izhodišče}, številske premice pa so \textbf{koordinatne osi}: $x$-os, $y$-os, $z$-os. \\

Dogovor: uporabljamo pozitivno orientiran koordinatni sistem. Če iz enote na $z$-osi pogledamo na $xy$-ravnine (to je ravnina, določena z $x$-osjo in $y$-osjo), vidimo pozitivno orientiran koordinatni sistem v ravnini. Če $x$-os zavrtimo za pozitivni kot 90 stopinj, dobimo $y$-os. \\

Definirajmo preslikavo iz $\mathbb{R}$ v prostor. Točko $T$ dobimo tako, da gremo od izhodišča po $x$-osi za $x$, po premici vzporedni $y$-osi za $y$ in po premici vzporedni $z$-osi za $z$. Števila $x,y,z$ v urejeni trojici $(x,y,z)$ so enolično določene s točko $T$. Npr. $z$ dobimo kot presečišče $z$-osi z ravnino skozi točko $T$, ki je vzporedna $xy$-ravnini. To pomeni, da je konstruirana preslikava, ki trojici $(x,y,z)$ priredi točko $T$, bijekcija iz $\mathbb{R}$ v prostor. Prostor z danim \linebreak koordinatnim sistemom zato identificiramo z $\mathbb{R}^3$. Pišemo $T(x,y,z)$ in številom $x,y,z$ pravimo \textbf{koordinate} točke $T$. \\

\textbf{Razdalja} med $T_1(x_1,y_1,z_1)$ in $T_2(x_2,y_2,z_2)$: $$d(T_1,T_2) = \sqrt[2]{(x_2 - x_1)^2 + (y_2 - y_1)^2 + (z_2 - z_1)^2}$$

Na $\mathbb{R}^3$ imamo dve operaciji:

\begin{itemize}
\item Seštevanje po komponentah: $$(x_1,y_1,z_1) + (x_2,y_2,z_2) = (x_1 + x_2, y_1 + y_2, z_1 + z_2)$$
\item Množenje s skalarji (notranja operacija): $$\lambda \in \mathbb{R} ,  (x,y,z) \in \mathbb{R} \Rightarrow \lambda (x,y,z) = (\lambda x, \lambda y, \lambda z)$$ 
Množenje s skalarji je definirano po komponentah.
\end{itemize}

\newpage

\begin{center}
\subsection{Vektorji}
\end{center}

V fiziki pogosto uporabljamo količine, ki imajo poleg velikosti tudi smer (hitrost, sila $\ldots$ ). \\

\begin{definicija}
Naj bo $\vec{a}=(x,y,z) \in \mathbb{R}$ poljubna točka. \textbf{Krajevni vektor} točke $\vec{a}$ je \linebreak usmerjena daljica od koordinatnega izhodišča do točke $a$. Oznaka: $\vec{a} = (x,y,z)$.\\
\end{definicija}

Koordinate krajevnega vektorja so enake koordinatam končne točke. Zato $\mathbb{R}^3$ lahko \linebreak identificiramo z množico vseh krajevnih vektorjev v prostoru. \\

\begin{definicija}
\textbf{Vektor} $\vec{a} = (x,y,z)$ je množica vseh usmerjenih daljic, ki jih dobimo z \linebreak  vzporednim premikom krajevnega vektorja $\vec{a} = (x,y,z)$.\\
\end{definicija}

Opomba: Natančno bomo pogosto rekli, da je vektor usmerjena daljica. Dve usmerjeni daljici določata isti vektor, če sta vzporedni, enako dolgi in kažeta v isto smer. \linebreak  Nenatančno bomo usmerjeni daljici od točke $A$ do točke $B$ pogosto rekli kar \textbf{vektor} od $A$ do $B$ in jo označili z $\overrightarrow{AB}$. vsakemu krajevnemu vektorju ustreza natanko en vektor oz. vektor je enolično določen z ustreznim krajevnim vektorjem. Vemo pa, da je vsak krajevni \linebreak  vektor enolično določen s svojo končno točko in da lahko prostor identificiramo z množico $\mathbb{R}$. \linebreak  Geometrijski pomen množice $\mathbb{R}$ je množica vektorjev. \\

Na $\mathbb{R}^3$ smo definirali seštevanje in množenje s skalarji. Kako se to prenese na množico vektorjev?

\begin{itemize}
\item \textbf{Množenje s skalarjem}: Če je $\alpha$ pozitiven, je usmerjena daljica določena z vektorjem $\alpha \vec{a}$, vzporedno usmerjeni daljici, določeni z vektorjem $a$, $\infty$- krat daljša in kaže v isto smer. Če je $\alpha$ negativen, je usmerjena daljica, določena z vektorjem $\alpha \vec{a}$, vzporedna usmerjeni daljici, določeni z vektorjem $\vec{a}$, (-$\alpha$)-krat daljša in kaže v nasprotno smer. Če vzamemo dve enako dolgi usmerjeni daljici, ki kažeta v isto smer, in ju pomnožimo z istim skalarjem, spet dobimo enako dolgi, vzporedno usmerjeni daljici, ki kažeta v isto smer. Zato je množenje vektorja s skalarjem dobro definirano. S skalarjem lahko množimo kjerkoli v prostoru. \\

\item \textbf{Seštevanje}: Če imata vektorja $\vec{a_1}$ in $\vec{a_2}$ skupni začetek, vsaka $\vec{a_1} + \vec{a_2}$ poteka od \linebreak  skupnega začetka do četrtega oglišča paralelograma, določenega z usmerjenima daljicama $\vec{a_1}$ in $\vec{a_2}$. Če usmerjeni daljici, ki določata vektorja $\vec{a_1}$ in $\vec{a_2}$, vzporedno premaknemo, tako da imata skupni začetek, se tudi diagonala paralelograma vzporedno premakne. Zato je seštevanje vektorjev dobro definirano. Seštevamo lahko kjerkoli v prostoru. \\
\end{itemize}
\newpage
\textbf{Odštevanje vektorjev} definiramo s predpisom: $\vec{a} - \vec{b} = \vec{a} + (- \vec{b})$. Vektorje odštevamo po komponentah. 

\begin{itemize}
\item Vektor $\vec{O} = (0,0,0)$ imenujemo \textbf{ničelni vektor}. Določen je z vsako usmerjeno daljico, ki se začne in konča v isti točki.
\item Vektor $ - \vec{a} = (-x, -y, -z)$ imenujemo \textbf{nasprotni vektor} vektorja $\vec{a}=(x,y,z)$.
\item Če je $\vec{a}$ določen z usmerjeno daljico od $A$ do $B$, je $- \vec{a}$ določen z usmerjeno daljico od $B$ do $A$. Pišemo: $- \overrightarrow{AB} = \overrightarrow{BA}$.\\
\end{itemize}

\begin{center}
\subsubsection{Lastnosti računanja z vektorji}
\end{center}

\begin{itemize}
\item Komutativnost
\item Asociativnost
\item $\vec{O} + \vec{a} = \vec{a} + \vec{O} = \vec{a}$
\item $\vec{a} + (-\vec{a}) = -\vec{a} + \vec{a} = \vec{O}$
\item Distributivnost $(\alpha + \beta)\vec{a} = \alpha \vec{a} + \beta \vec{a}, \alpha (\vec{a} + \vec{b}) = \alpha \vec{a} + \alpha \vec{b}$
\item $(\alpha \beta ) \vec{a} = \alpha ( \beta \vec{a})$
\item  $ 1 \cdot \vec{a} = \vec{a}$\\
\end{itemize}

\begin{definicija}
Naj bodo $a_1, a_2, a_3, \ldots , a_n$ poljubni vektorji. Vsak vektor oblike \linebreak  $\alpha_1 \vec{a_1} + \alpha_2 \vec{a_2} + \ldots + \alpha_n \vec{a_n}$ kjer so $ \alpha_1, \ldots \alpha \in \mathbb{R} $, imenujemo \textbf{linearna kombinacija} vektorjev $a_1, \ldots , a_n$.\\
\end{definicija}

\begin{definicija}
Vektorji $a_1, \ldots , a_n$ so \textbf{linearno odvisni} kadar lahko vsaj enega od njih izrazimo kot linearno kombinacijo ostalih. Vektorji so \textbf{linearno neodvisni}, kadar niso linearno odvisni (vektorji so torej linearno neodvisni, kadar nobenega od njih ne moremo izraziti kot linearno kombinacijo ostalih). En vektor je linearno odvisen, kadar je ničelni vektor. Dva vektorja $\vec{a}$ in $\vec{b}$ sta linearno odvisna, kadar je $ \vec{a} = \alpha \vec{b}$ za nek $\alpha \in \mathbb{R}$ ali $\vec{b} = \beta \vec{a}$ za nek $\beta \in \mathbb{R}$. \\
\end{definicija}

Vektorja $\vec{a}$ in $\vec{b}$ sta linearno odvisna natanko takrat, ko pripadajoča krajevna vektorja ležita na isti premici (sta kolinearna). Ekvivalentno, poljubni usmerjeni daljici, ki določata vektorja $\vec{a}$ in $ \vec{b}$, sta vzporedni. \\

Naj bosta $\vec{a}$ in $\vec{b}$ linearno neodvisna krajevna vektorja. Vsak vektor oblike $\alpha \vec{a} + \beta \vec{b}$ leži v ravnini, ki jo določata $\vec{a}$ in $\vec{b}$. Velja tudi obratno: vsak vektor $\vec{c}$ na ravnini, določeni z $\vec{a}$ in $\vec{b}$, lahko zapišemo kot linearno kombinacijo $ \vec{c} = \alpha \vec{a} + \beta \vec{b}$. Kako: premica, ki gre skozi konec $\vec{c}$-ja in je vzporedna vektorju $\vec{b}$, seka premico, določeno z vektorjem $\vec{a}$, v natančno eni točki, ki jo določa krajevni vektor $\alpha \vec{a}$ za nek $\alpha \in \mathbb{R}$. Podobno definiramo vektor $ \beta \vec{b}$. Po definiciji seštevanja je  $ \vec{c} = \alpha \vec{a} + \beta \vec{b}$. Velja še več: $\alpha, \beta$ sta enolično določena s $\vec{c}$. Rečemo, da je  $ \vec{c} = \alpha \vec{a} + \beta \vec{b}$ in $\vec{a'} \neq \vec{a}$ ali $\vec{b'} \neq \vec{b}$. Če je $\vec{a'} \neq \vec{a}$ je $\vec{a} = \frac{\beta ' - \beta}{\alpha ' - \alpha} \vec{b}$. Če je $\beta ' \neq \beta$ pa je $\vec{b} = \frac{\alpha ' - \alpha}{\beta ' - \beta} \vec{a}$. V obeh primerih dobimo protislovje s tem, da sta $\vec{a}$ in $\vec{b}$ linearno neodvisna. \\

Ravnina, določena z linearno neodvisnima krajevnima vektorjema $\vec{a}$ in $\vec{b}$ je natanko množica vseh linearnih kombinacij $\alpha \vec{a} + \beta \vec{b}$, kjer sta $\alpha, \beta \in \mathbb{R}$. \\

\begin{definicija}
\textbf{Baza ravnine} je množica, sestavljena iz dveh linearno neodvisnih vektorjev.\\
\end{definicija}

Dokazali smo, da se da vsak vektor v ravnini na enoličen način zapisati kot linearni kombinaciji baznih elementov. Dokazali smo tudi, da so trije vektorji linearno odvisni natanko takrat, ko pripadajoči krajevni vektorji ležijo v isti ravnini. \\

\begin{definicija}
\textbf{Baza prostora} je množica, sestavljena iz treh linearno neodvisnih vektorjev.\\
\end{definicija}

\begin{trditev}
Naj bo $\{\vec{a},\vec{b},\vec{c}\}$ baza prostora. Potem lahko vsak vektor $\vec{x} \in \mathbb{R}^3$ zapišemo kot linearno kombinacijo $\vec{x} = \alpha \vec{a} + \beta \vec{b} + \gamma \vec{c}$. Pri tem so $\alpha, \beta, \gamma$ enolično določeni z vektorjem $\vec{x}$.\\
\end{trditev}

\begin{trditev}
 Vektorji $\vec{a}, \vec{b}, \vec{c}$ so linearno odvisni takrat, ko obstajajo $\alpha, \beta, \gamma \in \mathbb{R}$, ne vsi 0, da je $\alpha \vec{a} + \beta \vec{b} + \gamma \vec{c} = \vec{O}$. Vektorji $\vec{a}, \vec{b}, \vec{c}$ so torej linearno neodvisni, kadar velja sklep $\alpha \vec{a} \beta \vec{b} + \gamma \vec{c} = \vec{O} \Rightarrow \alpha = \beta = \gamma = 0$ (oziroma edina linearna kombinacija, ki je 0, je tista, pri katerem so vsi koeficienti enaki 0, tj. \textbf{trivialna linearna kombinacija}.\\
\end{trditev}

\begin{definicija}
\textbf{Skalarni produkt} vektorjev $\vec{a_1}=(x_1, y_1, z_1)$ in $\vec{a_2} = (x_2, y_2, z_2)$ je število (skalar). $$ \vec{a_1} \cdot \vec{a_2} = x_1 x_2 + y_1 y_2 + z_1 z_2$$ 
\end{definicija}

\begin{posl}
Če je $\vec{a} = (x,y,z)$ potem je $x= \vec{a} \cdot \vec{i}, y = \vec{a} \cdot \vec{j}, z = \vec{a} \cdot \vec{k}$.\\
\end{posl}

Lastnosti skalarnega produkta:
\begin{itemize}
\item Komutativnost: $\vec{a} \cdot \vec{b} = \vec{b} \cdot \vec{a}$ za vsaka $\vec{a}, \vec{b} \in \mathbb{R}^2$
\item Distributivnost: $\vec{a}(\vec{b} + \vec{c}) = \vec{a} \vec{b} + \vec{a} \vec{c}$  (in $(\vec{b} + \vec{c}) \vec{a} = \vec{b} \vec{a} + \vec{c} \vec{a})$ za vse $\vec{a}, \vec{b}, \vec{c} \in \mathbb{R}^3$
\item Homogenost: $(\alpha \vec{a}) \cdot \vec{b} = \vec{a} \cdot (\alpha \vec{b}) = \alpha \cdot (\vec{a} \vec{b})$ za vse $\vec{a}, \vec{b} \in \mathbb{R}^2, \alpha \in \mathbb{R}$
\item Pozitivna definitnost: $\vec{a} \cdot \vec{a} \geq 0$ za vsaka $\vec{a} \in \mathbb{R}^3$ in $\vec{a} \cdot \vec{a} = 0$ v primeru, ko je $\vec{a} = 0$.\\
\end{itemize}

\begin{definicija}
\textbf{Dolžina} ali norma vektorja $\vec{a}$ je število $||\vec{a}|| = |\vec{a}| = \sqrt{\vec{a} \cdot \vec{a}}$. $$ \vec{a} = (x,y,z)$$ $$|| \vec{a}|| = |\vec{a}| = \sqrt{\vec{a} \cdot \vec{a}} = \sqrt{x^2 + y^2 + z^2}$$
\end{definicija}

Običajno dobimo usmerjene daljice od $(0,0,0)$ do $(x,y,z)$.\\

\begin{izrek}
$\vec{a} \cdot \vec{b} = |\vec{a}| \cdot |\vec{b}| \cdot cos\varphi$, kjer je $\varphi$ kot med usmerjenima daljicama, ki določata vektorja $\vec{a}$ in $\vec{b}$ in imata skupni začetek.\\
\end{izrek}

Dogovor: Ničelni vektor je pravokoten na vsak vektor. \\

\begin{posl}
Če je $\vec{a}$ pravokoten na $\vec{b}$, potem velja, da je $\vec{a} \cdot \vec{b} = 0$. $$ \vec{a} \perp \vec{b} \Rightarrow \vec{a} \cdot \vec{b} = 0$$
\end{posl}


PRIMER:
\begin{itemize}
\item \textsc{Ploščina trikotnika v ravnini $z=0$.} \\

Imejmo vektorja $\vec{a_1} = (x_1, y_1, 0)$ in $\vec{a_2} = (x_2, y_2, 0)$. Zanima nas ploščina paralelograma, napetega na $\vec{a_1}$ in $\vec{a_2}$. Recimo, da par $ (\vec{a_1}, \vec{a_2})$ pozitivno orientiran. Vektor $\vec{a_1}$ zavrtimo za 90 stopinj v pozitivni smeri in dobljeni vektor označimo z $\vec{a_1'}$. 

$$ P = |\vec{a_1}| \cdot |\vec{a_2}| \cdot sin\varphi$$ 

Če je par $(\vec{a_1}, \vec{a_2})$ negativno orientiran, dobimo

$$ P = - | \vec{a_1} | \cdot | \vec{a_2}| \cdot sin\varphi $$

Naj bo $\Theta$ kot med vektorjema $\vec{a_2}$ in $\vec{a_1'}$. Potem je $\Theta = \frac{\pi}{2} - \varphi$ , kadar je $\Theta$ med 0 stopinj in $\frac{\pi}{2}$, oziroma bo $\Theta = \varphi - \frac{\pi}{2}$, kadar bo $\varphi$ med $\frac{\pi}{2}$ in $\pi$. V obeh primerih je $cos\Theta = cos(\frac{\pi}{2} - \varphi) = sin \varphi$. 

\begin{gather*}
P = |\vec{a_1}| \cdot |\vec{a_2}| \cdot cos\Theta = |\vec{a_1'}| \cdot |\vec{a_2}| \cdot cos\Theta = \vec{a_1'} \cdot \vec{a_2} \\
 \vec{a_1} = |\vec{a_1} | \cdot (cos\delta, sin\delta)\\
\vec{a_1'} = |\vec{a_1}| (cos(\delta + \frac{\pi}{2}), sin(\delta + \frac{\pi}{2})  = |\vec{a_1}| (- sin\delta, cos\delta) = (- y_1, x_1) \\
 \vec{a_1} = (x_1, y_1) \\
 P = (- y_1, x_1)(x_2,y_2) = - x_2 y_1 + x_1 y_2 
\end{gather*}

Če je par $(\vec{a_1}, \vec{a_2})$ negativno orientiran, je $ P = x_2 y_1 - x_1 y_2 $ . 
\\ 
\\
Izraz $x_2 y_1 - x_1 y_2 $ nam pove produkt ploščine in orientacije paralelograma, napetega na $\vec{a_1} = (x_1, y_1)$ in $\vec{a_2} = (x_2, y_2)$. Orientacija je $+1$, če je par $(a_1,a_2)$ pozitivno orientiran, in je $-1$, če je negativno orientiran. \\

Izrazu $x_1 y_2 - x_2 y_1$ pravimo \textbf{$2 \times 2$ determinanta} in ga označimo: 
$$
\begin{vmatrix}
x_1 &  y_1  \\
x_2 & y_2 
\end{vmatrix} 
$$

Ugotovili smo, da bo determinanta enaka nič natanko takrat, kadar bosta vektorja $(x_1, y_1)$ in $(x_2, y_2)$ linearno odvisna. \\

\end{itemize}

\newpage 

\begin{center}
\subsection{Vektorski produkt}
\end{center}

\textbf{Motivacija}: $\vec{M} = \vec{r} \times \vec{F}.$ $|\vec{r}| \cdot |\vec{F}| \cdot sin\varphi $ \\

\begin{definicija}
\textbf{Vektorski produkt} vektorjev $\vec{a}$ in $\vec{b}$ je vektor $\vec{a} \times \vec{b}$ z naslednjimi lastnostmi:
\begin{enumerate}
\item  $\vec{a} \times \vec{b} \perp \vec{a}$ in $\vec{a} \times \vec{b} \perp \vec{b}$ 
\item $ |\vec{a} \times \vec{b}| = |\vec{a}| \cdot |\vec{b}| \cdot sin\varphi$ kjer je $\varphi$ kot med usmerjenima daljicama, določenima z $\vec{a}$ in $\vec{b}$, ki imata skupen začetek (torej $ |\vec{a} \times \vec{b}|$ je ploščina paralelograma, napetega na krajevna vektorja $\vec{a}$ in $\vec{b}$).
\item $(\vec{a}, \vec{b}, \vec{a} \times \vec{b})$ je pozitivno urejena trojica (to pomeni, če iz vrha $ \vec{a} \times \vec{b}$ pogledamo na ravnino, določeno z $\vec{a}$ in $\vec{b}$, potem se $\vec{a}$ pri vrtenju za pozitiven kot med 0 in $\phi$ zavrti v večkratnik vetorja $\vec{b}$.
\end{enumerate}
\end{definicija}

Dogovor: Ničelni vektor je vzporeden vsakemu vektorju. \\

\begin{posl}
$ \vec{a} \times \vec{b} = 0 \Leftrightarrow \vec{a} || \vec{b}$ \\
\end{posl}

\begin{itemize}
\item Poiščimo formulo za vektorski produkt.
$$ \vec{a} = (a_1, a_2, a_3), ~ \vec{b} = (b_1, b_2, b_3), ~ \vec{a} \times \vec{b} = (c_1, c_2, c_3)$$
Izračunajmo $ \vec{c_3} $.
$$ \vec{c_3} = (\vec{a} \times \vec{b}) \cdot \vec{k} $$
Naj bo $\vartheta$ kot med $\vec{a} \times \vec{b}$ in $\vec{k}$. Potem je 
$$ \vec{c_3} = | \vec{a} \times \vec{b} | \cdot |\vec{k}| \cdot cos\vartheta $$
(ta enačba predstavlja ploščino paralelograma napetega na $\vec{a}$ in $\vec{b}$).\\

Vektorja $\vec{a}$ in $ \vec{b}$ projiciramo na ravnino $z=0$ in dobljena vektorja označimo z $\vec{a'}$ in $\vec{b'}$.\\

Paralelogram, ki ga določata vektorja, ima ogliša $(0,0,0), (a_1, a_2, a_3), (b_1, b_2, b_3)$ in $ (a_1 + b_1, a_2 + b_2, a_3 + b_3)$. Projekcije teh točk so $(0,0,0), (a_1, a_2, 0), (b_1, b_2, 0)$ in $ (a_1 + b_1, a_2 + b_2, 0)$, ki spet tvorijo oglišča paralelograma. Zanima nas zveza med ploščinama paralelogramov $OBCA$ in $OB'C'A'$. \\

\ldots na to-do listi \ldots 
\end{itemize}

\newpage

\begin{center}
\subsubsection{Lastnosti vektorskega produkta}
\end{center}

\begin{enumerate}
\item Antikomutativnost: $ \vec{a} \times \vec{b} = - \vec{b} \times \vec{a}$ za vsaka $\vec{a}, \vec{b} \in \mathbb{R}^3$
\item Distributivnost: $ \vec{a} \times (\vec{b} + \vec{c}) = \vec{a} \times \vec{b} + \vec{a} \times \vec{c}$ in $(\vec{b} + \vec{c}) \times \vec{a} = \vec{b} \times \vec{a} + \vec{c} \times \vec{a}$ za vse $ \vec{a}, \vec{b}, \vec{c} \in \mathbb{R}^3$
\item Homogenost: $(\alpha \vec{a}) \times \vec{b} = \alpha ( \vec{a} \times \vec{b}) $ za vse $\vec{a}, \vec{b} \in \mathbb{R}^3$ in $\alpha \in \mathbb{R}$
\end{enumerate}

\newpage

\begin{center}
\subsection{Mešani produkt}
\end{center}

Mešan produkt vektorjev $\vec{a}$, $\vec{b}$ in $\vec{c}$ je število $[\vec{a},\vec{b},\vec{c}] = (\vec{a} \times \vec{b}) \cdot \vec{c}$. \\

$$ \vec{a} = (a_1, a_2, a_3),~ \vec{b} = (b_1, b_2, b_3),~ \vec{a} \times \vec{b} = (c_1, c_2, c_3)$$

$$
(\vec{a} \times \vec{b}) \cdot \vec{c}= (
\begin{vmatrix}
a_2 & a_3 \\
b_2 & b_3
\end{vmatrix}
,-
\begin{vmatrix}
a_1 & a_3 \\
b_1 & b_3
\end{vmatrix}
,
\begin{vmatrix}
a_1 & a_2 \\
b_1 & b_2 
\end{vmatrix}
) \cdot (c_1, c_2, c_3) = $$

$$ = c_1 \cdot
\begin{vmatrix}
a_2 & a_3 \\
b_2 & b_3
\end{vmatrix}
- c_2 \cdot
\begin{vmatrix}
a_1 & a_2 \\
b_1 & b_2
\end{vmatrix}
+ c_3 \cdot
\begin{vmatrix}
a_1 & a_2 \\
b_1 & b_2
\end{vmatrix}
= $$
$$ =
\begin{vmatrix}
c_1 & c_2 & c_3 \\
a_1 & a_2 & a_3 \\
b_1 & b_2 & b_3
\end{vmatrix} 
$$ \\

Geometrijska interpretacija: Paralelepiped je geometrijsko telo s tremi četvericami paroma vzporednih robov (``nagnjen kvader''). Določen je s tremi linearno neodvisnimi vektorji. Paralelepiped je poseben primer prizme, zato je $ V = p \times v $. \\

Naj bo paralelepiped določen z vektorji $\vec{a}, \vec{b}, \vec{c}$ in naj bo $\varphi$ kot med $\vec{c}$ in $\vec{a} \times \vec{b}$. $\varphi$ je tudi kot med vektorjem $\vec{c}$ in višino. 

$$ v = |\vec{c}| \cdot cos\varphi $$

$$ V = p \cdot v = |\vec{a} \times \vec{b}| \cdot |\vec{c}| \cdot |cos\varphi| = | ( \vec{a} \times \vec{b}) \cdot \vec{c}| = | [ \vec{a}, \vec{b}, \vec{c}]| $$ \\

Prostornina paralelepipeda je enaka absolutni vrednosti mešanega produkta. Predznak mešanega produkta nam pove orientacijo $[ \vec{a}, \vec{b}, \vec{c}]\geq 0 \Leftrightarrow [ \vec{a}, \vec{b}, \vec{c}] $ je pozitivno orientirana trojica ($\vec{c}$ ~ ``kaže gor'' $ \Leftrightarrow \varphi \in [0, \frac{\pi}{2}])$. \\

\newpage

\begin{center}
\subsubsection{Lastnosti mešanega produkta}
\end{center}

\begin{enumerate}
\item $[ \vec{a}, \vec{b}, \vec{c}] = [ \vec{b},\vec{c},\vec{a}] = - [ \vec{a}, \vec{c},\vec{b} ] $ za vse $\vec{a}, \vec{b}, \vec{c} \in \mathbb{R}^3$
\item Homogenost v vseh treh faktorjih: $\alpha [ \vec{a}, \vec{b}, \vec{c}] = [ \alpha \vec{a}, \vec{b}, \vec{c}] = [ \vec{a}, \alpha \vec{b}, \vec{c}] = [ \vec{a}, \vec{b}, \alpha \vec{c}] $
\item Distributivnost v vseh treh faktorjih
\end{enumerate}

\newpage

\begin{center}
\subsection{Dvojni vektorski produkt}
\end{center}

\ldots na to-do listi also \ldots 

\newpage

\begin{center}
\section{Enačbe premic in ravnin v $\mathbb{R}^3$}
\end{center}

Imejmo ravnino $\Sigma$. Enačba ravnine $\Sigma$ je taka enačba v spremenljivkah $x,y,z$, da velja: \\

\begin{itemize}
\item Točka $T(a,b,c)$ leži na ravnini $\Sigma$ natanko tedaj, kadar trojica $(a,b,c)$ zadošča enačbi.\\
\end{itemize}

\begin{definicija}
\textbf{Normala ravnine} je vsak neničelni vektor, ki je pravokoten na ravnino.\\
\end{definicija}
 
Več ravnin ima lahko isto normalo. Ravnine, ki imajo isto normalo, so \textbf{vzporedne}. \\

Ravnina je natančno določena s svojo normalo in eno točko na njej. Ta točka ni enolično določena z ravnino (lahko vzamemo poljubno točko). Tudi normala ni enolično določena (lahko jo pomnožimo s poljubnim neničelnim skalarjem). \\

\begin{itemize}
\item Imejmo ravnino $\Sigma$ z normalno $n$ in neko točko $T_0$ s krajevnim vektorjem $\vec{r_0}$ na njej. Iščemo enačbo ravnine $\Sigma$. \\

Naj bo $\vec{r}$ krajevni vektor poljubne točke $T$. 
$$ T \in \Sigma \Leftrightarrow T_0 T = \vec{r} - \vec{r_0} \in \Sigma \Leftrightarrow \vec{r} - \vec{r_0} \perp (\vec{r} - \vec{r_0}) \cdot \vec{n} = 0 $$

$\vec{n}$ je vektor, ki je pravokoten na vse vektorje v ravnini. \\

$ (\vec{r} - \vec{r_0}) \cdot \vec{n} = 0 $ je enačba ravnine $\Sigma$, ki ji pravimo \textbf{vektorska enačba ravnine}. \\

\item Po komponentah 
$$((x,y,z) - (x_0, y_0, z_0)) \cdot (a,b,c) = 0 $$
$$ ax - a x_0 + by - b y_0 + cz - c z_0 = 0 $$

Označimo
$$ d = - a x_0 - b y_0 - c z_0 = - n \cdot r_0 $$

\begin{center}
(To je realno število, saj sta $\vec{r}$ in  $\vec{r_0}$ znana).
\end{center}

Dobili smo enačbo $$ax + by + cz + d = 0$$

\item Vsako ravnino lahko zapišemo kot rešitev linearne enačbe $$ax +by+cz+d=0$$

Velja tudi obratno. \\

\item Vsaka enačba oblike $ax+by+cz+d=0$, kjer $a,b,c$, niso vsi 0, določa neko ravnino. \\

Definirajmo $$\vec{n}=(a,b,c)$$

Zaradi simetrije lahko predpostavimo, da je $a\neq0$. 

Izberemo poljubna $y_0, z_0 \in \mathbb{R} $ in definirajmo $x_0 = \frac{b y_0 + c z_0 + d}{a} $ in $ r_0 = (x_0, y_0, z_0)$. \\

Enačba ravnine z normalo $n$, ki poteka skozi točko s krajevnim vektorjem $\vec{r_o}$ je $$ (\vec{r} - \vec{r_0})\cdot \vec{n} =0 \Leftrightarrow ax+by+cz=-a- \frac{b y_0 + c z_0 + d}{a} + b y_0 + cz_0 $$

To je naša prvotna enačba. \\

Enačbi $ax+by+cz+d=0$ po navadi rečemo \textbf{implicitna enačba ravnine}. \\

Enačba ravnine ni enolično določena, ker si lahko izberemo drugo normalo ali drugo točko na ravnini. Določena je do množenja z neničelnim skalarjem natančno. Pogosto je ugodno, če ima normala dolžino 1. Zato normalo naravnamo tako, da namesto $n$ vzamemo $\frac{n}{|\overline{n}|}$ kar je enotski vektor. $$ |\vec{n}|=\sqrt{a^2+b^2+c^2} $$

Dobimo enačbo $$\frac{ax}{\sqrt{a^2+b^2+c^2}} + \frac{by}{\sqrt{a^2+b^2+c^2}} + \frac{cz}{\sqrt{a^2+b^2+c^2}} + \frac{d}{\sqrt{a^2+b^2+c^2}} = 0 $$

Tej enačbi pravimo \textbf{normalna enačba ravnine}. Določena je do predznaka natančno. \\
\end{itemize}

\newpage

\begin{center}
\subsection{Razdalja do ravnine}
\end{center}

\begin{itemize}
\item Imejmo ravnino $\Sigma$ z enačbo $(\vec{r} - \vec{r_0}) \cdot \vec{n} = 0$, kjer je $\vec{r_0} = (x_0,y_0,z_0)$ in $\vec{n} = (a,b,c)$ in točko $T_1$ s krajevnim vektorjem $\vec{r_1}=(x_1,y_1,z_1)$. \\

Zanima nas razdalja med ravnino $\Sigma$ in točko $T_1$. \\

$T_0$ naj bo točka s krajevnim vektorjem $\vec{r_0}$. \\

Naj bo $d=- \vec{n} \cdot \vec{r_0}$. Potem je $ax+by+cz+d=0$ enačba ravnine. \\

Naj bo $\Delta$ razdalja med $T_1$ in $\Sigma$. \\

Naj bo $\varphi$ kot med vektorjem $\overrightarrow{T_0 T_1} = \vec{r_1} - \vec{r_0} $ in zveznico med $T_1$ in pravokotno projekcijo $T_1$ na $\Sigma$. Potem je $\Delta = |\vec{r_1} - \vec{r_0}| \cdot cos\varphi $. \\

$\varphi$ je tudi kot med $\vec{n}$ in $\vec{r_1} - \vec{r_0} $, če $T_1$ leži v istem polprostoru, ki ga določa ravnina $\Sigma$, kot kaže normala. \\

Če je $T_1$ na drugi strani ravnine, je $T-\varphi$ kot med $\vec{n}$ in $\vec{r_1}-\vec{r_0}$. \\

Če je $T_1$ na isti strani ravnine kot normala, je $$(\vec{r_1}-\vec{r_0}) \cdot \vec{n}=0= |\vec{n}|\cdot|\vec{r_1}-\vec{r_0}|\cdot cos\varphi$$

Če je $T_1$ na drugi strani ravnine kot normala, pa je $$(\vec{r_1}-\vec{r_0}) \cdot \vec{n}= |\vec{n}|\cdot|\vec{r_1}-\vec{r_0}|\cdot cos(\pi - \varphi)=-|\vec{n}|\cdot|\vec{r_1}-\vec{r_0}|\cdot cos\varphi$$

V vsakem primeru je 
\begin{multline*}
|\vec{n}|\cdot|\vec{r_1}-\vec{r_0}|\cdot cos\varphi  = |\vec{n}\cdot(\vec{r_1}-\vec{r_0})|\Rightarrow \\ \Delta = |\vec{r_1}-\vec{r_0}|\cdot cos\varphi = \frac{|\vec{n}\cdot(\vec{r_1}-\vec{r_0})|}{|\vec{n}|}  \\ \Delta = \frac{|(\vec{n}\cdot(\vec{r_1}-\vec{r_0})|}{|\vec{n}|}
\end{multline*}

\item Predznak skalarnega produkta $\vec{n}\cdot(\vec{r_1}-\vec{r_0})$ nam pove, na kateri strani ravnine leži točka $T_1$: Če je predznak pozitiven, $T_1$ leži na tisti strani, kamor kaže normala. Če pa je predznak negativen, potem točka $T_1$ leži na drugi strani.\\

\item Po komponentah: 

\begin{align*}
\frac{|\vec{n}\cdot(\vec{r_1}-\vec{r_0})|}{|\vec{n}|} & = \frac{|(a,b,c)\cdot(x_1-x_0,y_1-y_0,z_1-z_0)|}{\sqrt{a^2+b^2+c^2}} \\
& = \frac{|a x_1 + b y_1 + c z_1 - (a x_0 + b y_0 + c z_0)|}{\sqrt{a^2+b^2+c^2}} \\
& = \frac{|a x_1 + b y_1 + c z_1 + d|}{\sqrt{a^2+b^2+c^2}}
\end{align*}

\item Točka $T_1$ leži v ravnini $\Leftrightarrow \Sigma =0$ \\

Razdalja med dvema ravninama je najkrajša razdalja med točko na prvi ravnini in točko na drugi ravnini. Če se ravnini sekata, je ta razdalja enaka nič. Če sta ravnini vzporedni, pa je razdalja med njima enaka razdalji med poljubno točko na prvi ravnini in drugo ravnino. \\

Razdalja med ravnino in premico, ki jo seka, je enaka 0. \\

Razdalja med ravnino in njej vzporedno premico pa je enaka razdalji med poljubno točko na premici in ravnino. \\

\end{itemize}

\newpage

\begin{center}
\subsection{Enačba premice}
\end{center}

Premico lahko gledamo kot presek dveh ravnin. Enačba premice $p$ bo zato sistem dveh linearnih enačb v spremenljivkah $x,y,z$ tako, da velja: točka $T(a,b,c,)$ leži na premici $p \Leftrightarrow $ trojica $ (a,b,c)$ ustreza obema enačbama.\\

\begin{definicija}
\textbf{Smerni vektor premice} $p$ je vsak neničelni vektor, ki je vzporeden premici $p$. \\
\end{definicija}

Premica je enolično določena z nekim svojim smernim vektorjem in neko točko na njej. \\

Smerni vektor ni enolično določen s premico, ampak ga lahko pomnožimo s poljubnim neničelnim skalarjem. \\

\begin{itemize}
\item Naj bo $p$ premica s smernim vektorjem $\vec{s}$ in točko $T_0$ na njej. $\vec{s}=(a,b,c)$, točka $T_0$ pa naj ima krajevni vektor $\vec{r_0} =(x_0,y_0,z_0)$. Naj bo $\vec{r}=(x,y,z)$ krajevni vektor poljubne točke $T$. Enačba premice $p$ bo tako enačba, ki bo veljala natanko tedaj, kadar bo točka $T_0$ ležala na premici $p$.

$$ T \in p \Leftrightarrow \overrightarrow{T_0 T} ~ || ~ s \Leftrightarrow \overrightarrow{T_0 T} = \vec{r} - \vec{r_0} = \lambda s, ~ \lambda \in \mathbb{R} $$

$ \vec{r} = \vec{r_0} + \lambda s, ~ \lambda \in \mathbb{R} $ je \textbf{vektorska parametrična enačba premice}.\\

\item Po komponentah

$$ (x,y,z)=(x_0,y_0,z_0)+\lambda(a,b,c) $$

$$ x=x_0 + \lambda a $$
$$ y=y_0 + \lambda b $$
$$ z=z_0 + \lambda c $$

\begin{center}
To je \textbf{parametrična enačba premice} ($\lambda$ je parameter).\\
\end{center}


\item Znebimo se parametra. Če je $abc\neq 0$, je $\lambda=\frac{x-x_0}{a}=\frac{y-y_0}{b}=\frac{z-z_0}{c} $
$$ \lambda = \frac{x-x_0}{a}=\frac{y-y_0}{b}=\frac{z-z_0}{c} $$

\begin{center}
To je \textbf{enačba premice}.
\end{center}

\item Če je npr. $a=0$ in $bc\neq0$, je $x=x_0$ in $\lambda=\frac{y-y_0}{b}=\frac{z-z_0}{c}$. V tem primeru je enačba premice: $x=x_0, \frac{y-y_0}{b}=\frac{z-z_0}{c}$. Če je npr. $a=b$, je $c\neq0$ in je enačba premice enaka $x=x_0,y=y_0$. \\

Premica je enolično določena z dvema točkama na njej. 

\item Naj bosta $A$ in $B$ točki na premici $p$ in naj bosta $\vec{r_A}$ in $\vec{r_B}$ njuna krajevna vektorja. Za točko na premici si lahko vzamemo $A$, za smerni vektor pa $\overrightarrow{AB}=\vec{r_B} - \vec{r_a}$. \textbf{Vektorska enačba premice} je torej $$ \vec{r}=\vec{r_A}+\lambda(\vec{r_B} - \vec{r_A}),~\lambda \in \mathbb{R}$$

\end{itemize}

\newpage

\begin{center}
\subsection{Razdalja do premice}
\end{center}

\begin{itemize}
\item Imejmo premico $p$ z enačbo $\vec{r}=\vec{r_0}+\lambda s,~\lambda \in \mathbb{R}$, kjer je $\vec{r_0}=(x_0,y_0,z_0)$ in $\vec{s}=(a,b,c)$. \\

Naj bo $T_1$ poljubna točka s krajevnim vektorjem $\vec{r_1}=(x_1,y_1,z_1)$. Zanima nas razdalja $\Delta$ med točko $T_1$ in premico $p$. \\

Najbližja točka točki $T_1$, ki leži na premici $p$, je pravokotna projekcija točki $T_1$ na $p$. \\

Naj bo $\varphi$ kot med vektorjema $\vec{s}$ in $\overrightarrow{T_0 T_1}=(\vec{r_1} - \vec{r_0})$. \\
$$ \Delta =|\overrightarrow{T_0 T_1}|\cdot sin\varphi ~~(=|\overrightarrow{T_0 T_1}| \cdot sin(\pi - \varphi)) $$

Vemo, da je $|\vec{s} \times (\vec{r_1} - \vec{r_0})|=|\vec{s}| \cdot |\vec{r_1}-\vec{r_0}|\cdot sin\varphi $ 

$$ \Delta = |\vec{r_1}-\vec{r_0}|\cdot sin\varphi = \frac{|\vec{s}\times(\vec{r_1}-\vec{r_0})|}{|\vec{s}|} $$ \\


PRIMER: 
\begin{itemize}
\item[$\bullet$] \textit{Izračunaj razdaljo med točko $A(1,1,1)$ in premico z enačbo  \linebreak $\frac{x}{2}={y+1}{2}=z=1$} \\

\begin{enumerate}
\item $\vec{r_1}=(1,1,1)$, $\vec{r_0}=(0,-1,-1)$, $\vec{s}=(2,-2,1)$
\item $\vec{r_1}-\vec{r_0}=(1,2,2)$
\item $\vec{s}\times(\vec{r_1}-\vec{r_0})=(-6,-3,6)$
\item $\Delta=3$ \\
\end{enumerate}

Če se premici sekiata, je razdalja med njima enaka 0. 
Če sta premici vzporedni, je razdalja med njima enaka razdalji me poljubno točko na prvi premici in drugo premico.

\end{itemize}

\end{itemize}

\newpage

\begin{center}
\subsection{Razdalja med mimobežnima premicama}
\end{center}

\begin{itemize}
\item Imejmo premici $p_1$ in $p_2$ z enačbama $\vec{r}=\vec{r_1}+\lambda s,~\lambda \in \mathbb{R}$ in $\vec{r}=\vec{r_2}+\lambda s,~\lambda \in \mathbb{R}$.\\

Predpostavimo, da $s_1$ ni vzporedna z $s_2$.\\

$T_1$ naj bo točka s krajevnim vektorjem $\vec{r_1}$, $T_2$ pa točka s krajevnim vektorjem $\vec{r_2}$.  \linebreak $\Delta$ naj bo razdalja med $p_1$ in $p_2$ (najboljša razdalja med neko točko iz $p_1$ in neko točko iz $p_2$). \\

$q_2$ naj bo premica skozi $T_2$, ki je vzporedna $p_1$, $q_1$ pa naj bo premica skozi $T_1$, ki je vzporedna $p_1$. \\

Premici $p_1$ in $q_1$ določata ravnino $\Sigma_1$, premici $p_2$ in $q_2$ pa ravnino $\Sigma_2$. Ravnini sta vzporedni, saj obe vsebujeta nekolinearna vektorja $\vec{s_1}$ in $\vec{s_2}$. \\

$p_1 \in \Sigma_1, p_2 \in \Sigma_2 \Rightarrow \Delta$ je večja ali enaka razdalji med ravninama $\Sigma_1$ in $\Sigma_2$ (minimum po večji množici je manjši ali enak minimumu po manjši množici). \\

$Q_1$ naj bo presečišče $p_1$ in pravokotne projekcije $p_2$ na ravnino $\Sigma_1$. \\
$Q_2$ pa naj bo presečišče $p_2$ in pravokotne projekcije $p_1$ na ravnino $\Sigma_2$.

\begin{gather*}
|\overrightarrow{Q_1 Q_2}| \geq \Delta,~ po~konstrukciji~pa~je~|\overrightarrow{Q_1 Q_2}|=d(\Sigma_1,\Sigma_2) \\
\Delta \leq |\overrightarrow{Q_1 Q_2}|=d(\Sigma_1,\Sigma_2) \leq d(\Sigma_1,\Sigma_2) \leq \Delta \Rightarrow \\ 
\Delta = |\overrightarrow{Q_1 Q_2}|=d(\Sigma_1,\Sigma_2)=d(T_2, \Sigma_1)
\end{gather*}

Za normalo na ravnino $\Sigma_1$ lahko vzamemo vektorski produkt $\vec{s_1} \times \vec{s_2}$. \\

Torej je: 

$$ \Delta = \frac{|(\vec{r_2}-\vec{r_1})\cdot(\vec{s_1} \times \vec{s_2})|}{|\vec{s_1} \times \vec{s_2}|} = \frac{|[\vec{r_2}-\vec{r_1},\vec{s_1},\vec{s_2}]|}{|\vec{s_1} \times \vec{s_2}|} $$ \\

Premici $p_1$ in $p_2$ se sekata $\Leftrightarrow [\vec{r_2}-\vec{r_1},\vec{s_1},\vec{s_2}]=0$.

\end{itemize}


\newpage

\begin{center}
\section{Osnovne algebrske strukture}
\subsection{Ponovitev preslikav}
\end{center}

Preslikava $ f: A \to B $ je predpis, ki vsakemu elementu množice $A$ priredi natanko en element množice $B$. Elementu $a \in A$ priredimo element iz B, ki ga označimo z $f(a)$ in ga imenujemo \textbf{slika elementa} $a$. \\

Preslikavo preprosto imenujemo tudi funkcija, predvsem v primerih, če je $B$  množica števil. Množico $A$ imenujemo \textbf{domena} ali \textbf{definicijsko območje} preslikave, množico $B$ pa \textbf{kodomena} ali \textbf{zaloga vrednosti} preslikave. \\

\textbf{Zaloga vrednosti preslikave} $f: A \rightarrow B$ je množica $Z_f = \{f(x), x \in A\}$ 

Preslikavo določajo domena, kodomena in funkcijski predpis.\\

Če imata dve preslikavi isti predpis, domeni ali kodomeni pa sta različni, sta preslikavi različni. \\

Naj bo $C \subseteq A$ in $D \subseteq B$ in predpostavimo, da je $f(x) \in D$ za nek $x \in C$. Potem lahko definiramo preslikavo $g: C \to D$ s predpisom $g(x) = f(x)$ za $x \in C$. Preslikavi g, določeni s tem predpisom, pravimo \textbf{zožitev preslikave $f$ na množico} $C$ in pišemo $g = f|_c$. Namesto $f|_c$ bomo mogoče včasih pisali kar $f$ in takrat bo moralo biti iz konteksta jasno, da gre za zožitev. Preslikavi $f$ pravimo \textbf{razširitev preslikave $g$}. \\

Preslikava $f: A \to B$ je \textbf{surjektivna}, kadar je $Z_f = B$. To pomeni, da je vsak element množice $B$ slika nekega elementa iz $A$. \\

Preslikava $f: A \to B$ je \textbf{injektivna}, kadar za vsaka različna elementa $x,y \in A$ velja \linebreak $f(x) \neq f(y)$. Ekvivalentno, $f$ je injektivna, kadar velja implikacija $x,y \in A. f(x) = f(y) \Rightarrow x = y$. \\

Preslikava $f: A \to B$ je \textbf{bijektivna}, kadar je surjektivna in injektivna hkrati. \\

Množici $A$ in $B$ imata \textbf{isto moč}, kadar obstaja bijektivna preslikava med njima. \\

Če je preslikava $f: A \to B$ bijektivna, potem za vsak $y \in B$ obstaja enoličen $x \in A$, da je $f(x) = y$. Preslikava $G: B \to A$, definirana s predpisom $g(y) =$ tisti $x \in A$, za katerega je $f(x) = y$, se imenujemo \textbf{inverzna preslikava} preslikave $f$. Oznaka: $f^{-1}$. $f^{-1}$ je bijektivna. \\

Naj bo $C \subseteq B$. Množico $f^{-1}(C)=\{x \in A ; f(x) \in C \}$ imenujemo \textbf{praslika množice} $C$. $$f^{-1}(C) \subseteq A$$ \\

Če je $C=\{y\}$, potem pišemo $f^{-1}(y)$ namesto $f^{-1}(\{y~\})$ in tej množici rečemo \textbf{praslika elementa} $y$. Oznaka $f^{-1}(y)$ ne pomeni, da ima preslikava inverz. \\

Oznaka za sliko množice $C \subseteq A$: $f(C) = \{f(x), x \in C \} \subseteq B$ \\

\textbf{Kompozitum preslikav} $f: A \to B$ in $g: B \to C$ je preslikava $g \circ f : A \to C$ \linebreak definirana s predpisom $(g \circ f)(x) = g(f(x))$ za vsak $x \in A$. Ekvivalentno, to je tista preslikava $g \circ f: A \to C$, za katero komutira diagram (če gremo na katerikoli način v smeri puščic, dobimo isti rezultat). \\

\textbf{Identiteta množice} $A$ je preslikava $id_A: A \to A$, definirana s predpisom $id_A(x) = x$ za vsak $x \in A$. \\

\begin{itemize}
\item $f: A \to B \Rightarrow id_B \circ f, f \circ id_A = f$
\item Če je $f: A \to B$ bijektivna, potem je $f \circ f^{-1} = id_B, f^{-1} \circ f = id_A$\\
\end{itemize}

\begin{trditev}
Naj bosta $f: A \to B$ in $g: B \to C$ preslikavi. Potem velja: 
\begin{itemize}
\item[1)] Če sta $f$ in $g$ injektivni, je $g \circ f$ injektiven
\item[2)] Če sta $f$ in $g$ surjektivni, je $g \circ f$ surjektiven
\item[3)] Če sta f in g bijektivni, je $g \circ f$ bijektiven in $(g \circ f)^{-1} = f^{-1} \circ g^{-1} $
\item[4)] Če je $g \circ f $ injektivna, je f injektivna 
\item[5)] Če je $ g \circ f $ surjektivna, je g surjektivna \\
\end{itemize}
\end{trditev}

\begin{posl}
$f: A \to B$ je bijektivna preslikava z inverzom $g: B \to A \Leftrightarrow f \circ g = id_B$ in $g \circ f = id_A$. V tem primeru je $g=f$. \\
\end{posl}

\textbf{Graf preslikave} $f: A \to B$ je množica $\Gamma(f) = \{(x,f(x)).x\in A \} \subseteq A \times B\} $

\newpage

\begin{center}
\subsection{Operacije}
\end{center}

\noindent \textbf{Operacija} na množici $A$ je preslikava $A \times A \to A$, kjer  $(x,y) \mapsto x \cdot y$. Natančneje, to je \textbf{dvočlena} (ali binarna) \textbf{notranja} operacija. Element $x \circ y$ običajno imenujemo \linebreak \textbf{kompozitum elementov} $x$ in $y$. \\

\noindent PRIMERI: 
\begin{enumerate}
\item $(\mathbb{N}, \circ)$ je množenje ali seštevanje. Odštevanje ni operacija na $\mathbb{N}$. Na primer $1-2=-1$ ni element naravnih števil.
\item $(\mathbb{R}, \circ)$ je množenje ali seštevanje ali odštevanje. Deljenje ni operacija na $\mathbb{R}$. $\frac{1}{0}$ ni realno število.
\item $(\mathbb{R}^3, \circ)$ je seštevanje vektorjev ali vektorski produkt. Skalarni produkt ni operacija na $\mathbb{R}$, ker rezultat ni vektor.
\item Naj bo $A$ neprazna množica in $F(A$ množica vseh preslikav $A \to A$. Na $F(A)$ lahko definiramo operacijo $(f \circ g) \mapsto f \circ g$ običajen kompozitum preslikav $(f \circ g)(x)=f(g(x))$ za vsak $ x \in A$
\end{enumerate}

\noindent $n$-člena (ali \textbf{$n$-terna}) operacija na množici $A$ je preslikava 
$$\underbrace{A \times A \times A \times ... \times A}_{n} \to A$$ Preslikava $A \to A$ je \textbf{enočlena} operacija. \\

\noindent PRIMERI:
\begin{itemize}
\item Naslednik je enočlena operacija na $\mathbb{N}$.
\item Nasprotno število je enočlena operacija na $\mathbb{Z}, \mathbb{Q}, \mathbb{R}$.
\item $\mathbb{R} \times \mathbb{R} \times \mathbb{R} \rightarrow \mathbb{R},~(a,b,c) \mapsto ab+c$ je tročlena operacija.
\item Težišče treh točk v ravnini je tročlena operacija na $\mathbb{R}^2$.\\
\end{itemize}

\noindent Naj bosta $A$ in $R$ dve množici. \textbf{Zunanja} (linearna) \textbf{operacija} na množici $A$ je preslikava $R \times A \to A$. \\


\noindent PRIMER:
\begin{itemize}
\item Množenje vektorja s skalarjem je preslikava 
\begin{eqnarray*} 
\mathbb{R} \times \mathbb{R} \to \mathbb{R}^3 \\
(\alpha,\vec{x})\mapsto \alpha \vec{x}
\end{eqnarray*}
Torej je to zunanja binarna operacija na $\mathbb{R}^3$.
\end{itemize}

\noindent Če je na množici $A$ definirana vsaj ena notranja zunanja operacija, pravimo, da ima množica $A$ \textbf{algebraično strukturo}. \\

\noindent Obravnavali bomo samo linearne operacije, najprej notranje, nato pa tudi zunanje. \\

\noindent \textbf{Nekatere lastnosti operacij}:
\begin{itemize}
\item Operacija $\circ$ na množici $A$ je \textbf{asociativna}, kadar velja $(a \circ b) \circ c = a \circ (b \circ c),~\forall a,b,c \in A$
\item Če je operacija asociativna, ima izraz $a \circ b \circ c$ smisel, saj ni odvisen od tega, kako postavimo oklepaje.\\
\end{itemize}

\begin{trditev}
Če je operacija $\circ$ na množici $A$ asociativna, potem je izraz 
$a_1 \circ a_2 \circ ... \circ a_n$, kjer so $a_1, ..., a_n \in A$, neodvisen od tega, kako postavimo oklepaje.\\
\end{trditev}

Brez dokaza, lahko bi dokazali z indukcijo na $n$. \\

\begin{posl}
Če je operacija $\circ$ asociativna, je izraz $a_1 \circ a_2 \circ ... \circ a_n$ dobro definiran.\\
\end{posl}

\noindent Pri asociativnih operacijah oklepaje po navadi izpuščamo.\\

\noindent Operacija na množici $A$ je \textbf{komutativna} kadar velja $a \circ b = b \circ a,~\forall a,b\in A$. Če za elementa $a$ in $b$ velja $a \circ b = b \circ a$, pravimo, da elementa komutirata. \\

\noindent PRIMERI:
\begin{itemize}
\item Seštevanje in množenje števil sta komutativni in asociativni operaciji.
\item Odštevanje ni komutativno in ni asociativno. $$(a-(b-c)\neq(a-b)-c)$$
\item Vektorski produkt ni komutativen in  ni asociativen. $$(\vec{a} \times \vec{b})\times \vec{c} \neq \vec{a} \times (\vec{b}\times \vec{c})$$
\item Komponiranje funkcij je asociativno in v splošnem ni komutativno.\\
\end{itemize}

\noindent Naj bo $\circ$ operacija na množici $A$. Element $e\in A$ se imenuje \textbf{enota} ali \textbf{nevtralni element}, kadar velja $e \circ a = a \circ e = a,~\forall a \in A$. Splošneje, če velja $e \circ a = a,~\forall a \in A$, elementu $e$ pravimo \textbf{leva enota}, če velja $a \circ e = a,~\forall a \in A$, pa elementu $e$ pravimo \textbf{desna enota} \\

\newpage 

\noindent PRIMERI:
\begin{itemize}
\item $(\mathbb{R},+):~0$ je enota
\item $(\mathbb{R}^3, \times)$: ni enote
\item $(\mathbb{R},\cdot):~1$ je enota
\item $(F(A),\circ):~id_A$ je enota
\item $(\mathbb{C} \times \mathbb{C} \to \mathbb{C},~(z,w) \mapsto z|w|$ vsako število po absolutni vrednosti enako 1, je desna enota.\\
\end{itemize}

\begin{trditev}
Če obstajata leva enota $e$ in desna enota $f$, potem sta enaki in $e=f$ je \linebreak obojestranska enota.\\
\end{trditev}

\begin{proof}
$e=ef=f$ ($f$ je desna enota, $e$ je leva enota)\\
\end{proof}

\begin{posl}
Če obstaja (obojestranska) enota, je ena sama.\\
\end{posl}

\noindent Naj bo $\circ$ operacija na množici $A$, ki ima enoto $e\in A$. Naj bo $a \in A$. Če obstaja element $a\in A$, da je $a \circ a' =e$, potem elementu $a'$ pravimo \textbf{desni inverz} elementa $a$. Če obstaja element $a'' \in A$, da je $a'' \circ a = e$, potem elementu $a''$ pravimo \textbf{levi inverz} elementa $a$. Če obstaja element $a''' \in A$, ki je hkrati levi in desni inverz elementa $a$, mu pravimo \textbf{inverz} elementa $a$ in ga označimo z $a^{-1}$. \\

\noindent PRIMERI:
\begin{itemize}
\item $(\mathbb{R},+):~-a$ je inverz od $a$
\item $(\mathbb{R},\cdot):~\frac{1}{a}$ je inverz od $a$, obstaja le za $a\neq 0$
\item $F(A)$: inverzi v splošnem ne obstajajo, $f\in F(A)$ ima inverz natanko tedaj, kadar je $f$ bijektivna (in inverz je v tem primeru običajen inverz funkcije)\\
\end{itemize}

\begin{trditev}
Naj bo $\circ$ asociativna operacija in naj bo $a'$ levi inverz elementa $a$, $a''$ pa naj bo desni inverz elementa $a$. Potem je $a''=a'$ in to je inverz elementa $a$.\\
\end{trditev}

\begin{proof}
$a'=a'\circ e=a'\circ(a\circ a'')=(a'\circ a)\circ a''=e\circ a''=a''$\\
\end{proof}

\begin{posl}
Če je $\circ$ asociativna operacija na množici $A$ in obstaja inverz elementa $a \in A$, potem je ta inverz en sam.\\
\end{posl}

\newpage

\begin{center}
\subsection{Grupe}
\end{center}

\noindent \textbf{Grupoid} je neprazna množica z notranjo binarno operacijo. \\

\noindent Grupoid, v katerem je operacija asociativna, se imenuje \textbf{polgrupa}. \\

\noindent \textbf{Monoid} je polgrupa, v kateri obstaja enota za operacijo. \\

\noindent Grupoid, polgrupa, monoid so določeni z množico $A$ in opracijo $\circ$ na njej. Pišemo $(A,\circ)$. Če bo jasno, za katero operacijo gre, bomo pogosto namesto $(A,\circ)$ pisali kar $A$. \\

\noindent PRIMERI:
\begin{itemize}
\item $(\mathbb{N},+), (\mathbb{Z},+), (\mathbb{Q},+), (\mathbb{R},+)$ so polgrupe. Razen $(\mathbb{N},+)$ so tudi monoidi, enota je 0
\item $(\mathbb{N},\cdot), (\mathbb{Z},\cdot), (\mathbb{Q},\cdot), (\mathbb{R},\cdot)$ so monoidi z enoto 1
\item $(\mathbb{Z},-)$ je samo grupoid, enako $(\mathbb{R}^3,\times)$
\item $(F(A),\cdot)$ je monoid z enoto $id_A$ \\
\end{itemize}

\begin{trditev}
Naj bo $(A, \circ)$ monoid z enoto $e$ in naj bosta $a,b\in A$ obrnljiva elementa. Potem je element $a \circ b$ tudi obrnljiv in velja $(a\circ b)^{-1}=b^{-1}\circ a^{-1}$. 
\end{trditev}

\noindent Obrnljiv element monoida je element, ki ima inverz. \\

\begin{dokaz}
\begin{align*}
(a\circ b)\circ (b^{-1}\circ a^{-1}) & = ((a\circ b)\circ b^{-1})\circ a^{-1}\\
& = (a\circ (b\circ b^{-1}))\circ a^{-1}\\
& =(a\circ e)\circ a^{-1} \\
& = a\circ a^{-1} \\
& = e 
\end{align*}
\begin{align*}
(b^{-1} \circ a^{-1})\circ (a\circ b) & =(b^{-1}\circ a^{-1})\circ a)\circ b \\
& =(b^{-1}\circ (a^{-1} \circ a))\circ b\\
& =(b^{-1}\circ e)\circ b  \\
& =b^{-1}\circ b \\
& =e
\end{align*}
Dokazali smo, da je $b^{-1}\circ a^{-1}$ res inverz elementa $a\circ b$.\\
\end{dokaz}

\begin{definicija}
\textbf{Grupa} je monoid, v katerem je vsak element obrnljiv.
\end{definicija}

\newpage 

\noindent Množica $G$ z operacijo $\circ$ je torej grupa, kadar:
\begin{enumerate}
\item Asociativnost: $(a\circ b)\circ c=a\circ(b\circ c),~\forall a,b,c\in G$
\item Obstaja (enoličen element) $e\in G$ (enota), da je $a\circ a=e\circ a=a,~\forall a\in G$
\item Za vsak $a\in G$ obstaja (enoličen) element $a^{-1}\in G$, da je \\
$a\circ a^{-1}=a^{-1}\circ a=e$. Pri tem je $a^{-1}$ inverz elementa $a$.\\
\end{enumerate}

\noindent PRIMERI: 
\begin{itemize}
\item $(\mathbb{C},+), (\mathbb{Z},+), (\mathbb{Q},+), (\mathbb{R},+)$ so grupe, enota je 0, inverz elementa $a$ je $-a$
\item $(\mathbb{C},\cdot), (\mathbb{Z},\cdot), (\mathbb{Q},\cdot), (\mathbb{R},\cdot)$ niso grupe, ker 0 nima inverza
\item $(\mathbb{C}\backslash \{ 0 \},\cdot), (\mathbb{Q}\backslash \{ 0 \},\cdot), (\mathbb{R}\backslash \{ 0 \},\cdot)$ so grupe, enota je 1, inverz od $a$ je $\frac{1}{a}$
\item $(\mathbb{Z}\backslash \{ 0 \},\cdot)$ ni grupa, ker npr. $\frac{1}{2} \notin \mathbb{Z}$
\item Naj bo $A\neq \emptyset$ neprazna množica in $F(A)$ množica preslikav $A\to A$. Ali je $(F(A),\circ)$ grupa? \\

Če ima A vsaj 2 elementa, $F(A)$ ni grupa. Naj bo $a\in A$ poljuben element in $f: A\to A$ preslikava, definirana s predpisom $f(x)=a$ za vsak $x\in A$. Dokažemo, da $f$ ni obrnljiv v $F(A)$.\\

Recimo, da obstaja $g\in F(A)$, da je $g\circ f=f \circ g=id_A$
$$f\circ g=id_A$$
$$f(g(x))=x,~\forall x \in A$$
$$ a=x,~\forall x\in A$$

To je protislovje, saj ima $A$ vsaj 2 elementa. Torej $F(A)$ ni grupa.

\item $A\neq \emptyset$. S $S(A)$ označimo množico vseh bijektivnih preslikav $A\to A$. \\

Vemo, da je kompozitum bijekcij bijekcija, zato je $\circ$ notranja operacija na $S(A)$. Vemo tudi, da je kompozitum asociativen, $id_A$ je enota za $\circ$, za inverzno preslikavo $f^{-1}$ pa velja $f\circ f^{-1}=f^{-1}\circ f=id_A$ in $f^{-1}\in S(A)$. Torej je $f^{-1}$ inverz elementa $f\in S(A) \Rightarrow S(A)$ je grupa.
\end{itemize}

\noindent V grupi običajno namesto $a\circ b$ pišemo kar $a\cdot b$ ali $ab$ in operciji rečemo \textit{produkt}. Definiramo tudi $a^n=\underbrace{a\cdot a \cdot \ldots \cdot a}_{n}$, če je $n\in \mathbb{N}$ in $a\in G,a^0=e$, $a^{-n}=(a^{-1})^n,~\forall n\in \mathbb{N},~a\in G$. Namesto $e$ običajno pišemo kar 1. \\

\noindent Pravimo, da uporabljamo \textit{multiplikativni zapis operacije} ($\cdot$ namesto $\circ$, 1 namesto $e$, potence $a^n$) \\

\noindent Grupo določa par $(G,\cdot)$ (oziroma $(G,\circ)$). Če je jasno, za katero operacijo gre, govorimo kar o grupi $G$. \\

\begin{trditev}
V grupi $G$ za vsak $a\in G$ in vsaka $m,n\in \mathbb{Z}$ velja $a^m \cdot a^n=a^{m+n}$ in $(a^m)^n=a^{mn}$.\\
\end{trditev}

\noindent Brez dokaza. Dokažimo lahko z indukcijo na $n$ za $n>0$, za $n<0$ pa upoštevamo definicijo inverza.\\

\begin{posl}
$a^{-n}=(a^n)^{-1}$\\
\end{posl}

\begin{definicija}
$G$ je \textbf{komutativna grupa} ali \textbf{Abelova grupa}, kadar velja $ab=ba$ za vsaka $a,b\in G$.\\
\end{definicija}

\noindent V Abelovih grupah običajno uporabljamo aditivni zapis: namesto $\cdot$  pišemo $+$ in operaciji rečemo \textit{seštevanje}, namesto 1 pišemo 0, namesto $a^{-1}$ pišemo $-a$, namesto $a^n$ pišemo $na$.\\

\begin{posl}
V komutativni grupi $G$ za vsaka $a,b\in G$ in vsaka $m,n\in \mathbb{Z}$ velja: \linebreak $ma+na=(m+n)a$, $n(ma)=mna$ in $n(a+b)=na+nb$. Če $G$ komutativna, ne velja nujno $(ab)^n=a^n \cdot b^n$\\
\end{posl}

\begin{trditev}
Naj bo $G$ grupa in $a,b,c\in G$ taki elementi, da je $ac=bc$ ali $ca=cb$. Potem je $a=b$.\\
\end{trditev}

\begin{proof}
Če je $ac=bc$, to enakost pomnožimo s $c^{-1}$ z desne $\Rightarrow a=b$. \\

Če je $ca=cb$ pa enakost pomnožimo z $c^{-1}$ z leve $\Rightarrow a=b$. \\

Pravimo, da v grupi lahko krajšamo. \\

A pozor: $ac=cb \neq a=b$.\\
\end{proof}

\newpage 

\noindent PRIMER:
\begin{itemize}
\item Če je $G$ le polgrupa in ni grupa, trditev ne velja. Primer: 

\begin{gather*}
G=F(A),~|A| \geq 2. ~ f: A \rightarrow A~ \text{je konstantna preslikava} \\
\Rightarrow f \circ g = f \circ h,~\forall g,h \in F(A) \\
f(g(x))=f(h(x)),~\forall x \in A\\
\end{gather*}

Pri Logiki in množicah boste videli, da z leve lahko krajšamo natanko z injektivnimi preslikavami, z desne pa s surjektivnimi.\\
\end{itemize}

\noindent Končne grupe pogosto podamo s \textbf{tabelo množenja}. V prvi stolpec in prvo vrstico napišemo vse elemente grupe $x_1,\ldots,x_n$. Na križišču i-te vrstice in j-tega stolpca napišemo $x_i x_j$. Iz prejšnje trditve sledi, da se v nobeni vrstici in v nobenem stolpcu element ne ponovi. Grupa je komutativna natanko tedaj, ko je tabela množenja simetrična glede na glavno diagonalo.

\begin{table}[hbp]
\centering
\label{tab:tmn}
\caption{Tabela množenja}
\begin{tabular}{c|ccccc}
$\cdot$ & $x_1$ & $x_2$ & $x_3$ & $\cdots$ & $x_n$ \\
\hline
$x_1$ & $1$ & $x_2$ & $\cdots$ & $\cdots$ & $x_n$ \\
$x_2$ & $x_2$ & $x_1^2$ & $\cdots$ & $\cdots$ & $x_2 x_n$ \\
$\cdots$ & $\cdots$ & $\cdots$ & $\cdots$ & $\cdots$ & $\cdots$ \\
$x_n$ & $x_n$ & $x_n x_2$ & $\cdots$ & $\cdots$ & $x_n^2$ \\
\end{tabular}
\end{table} 

\newpage 

\noindent PRIMERI: Poiščimo vse tabele množenja za grupe majhnih moči.
\begin{itemize}
\item $|G|=1$ Edina taka grupa je \textit{trivialna grupa} $\{ 1 \}$.
\item $|G|=2$

\begin{table}[hbp]
\begin{minipage}[b]{0.30\linewidth}\centering
\begin{tabular}{c|cc}
$\cdot$ & 1 & a \\
\hline
1 & 1 & a \\
a & a & 1 \\
\end{tabular}
\end{minipage}
\hspace{0.5cm}
\begin{minipage}[b]{0.30\linewidth}\centering
\begin{tabular}{c|cc}
$\Leftrightarrow$ & T & F \\
\hline
T & T & F \\
F & F & T \\
\end{tabular}
\end{minipage}
\hspace{0.5cm}
\begin{minipage}[b]{0.30\linewidth}\centering
\begin{tabular}{c|cc}
$+$ & 0 & 1 \\
\hline
0 & 0 & 1 \\
1 & 1 & 0 \\
\end{tabular}
\end{minipage}
\end{table}

Zadnja tabela skriva v sebi seštevanje po modolu 2. \\

Splošneje, naj bo $n>1$ naravno število. Z $\mathbb{Z}_n$ označimo množico ostankov celih števil pri deljenju z $n$. 
$$ \mathbb{Z}_n=\{ 0,1,\ldots,n-1 \}$$
V $\mathbb{Z}_n$ definiramo seštevanje s predpisom $a+b=\text{ostanek vsote števil $a$ in $b$}$ pri deljenju z $n$. \\

Na primer v $\mathbb{Z}_6$ je 3+5=2, ker da 8 ostanek 2 pri deljenju s 6. \\

Izkaže se, da je $(\mathbb{Z},+)$ grupa, enota je 0, inverz od $a$ je $n-a$. \\

``Edina'' grupa moči 2 je $\mathbb{Z}_2$. Natančneje, rekli bomo, da so vse grupe moči 2 izomorfne $\mathbb{Z}_2$. 

\item $|G|=3$

\begin{table}[hpb]
\begin{minipage}[b]{0.45\linewidth}
\centering
\begin{tabular}{c|ccc}
$\cdot$ & 1 & a & b \\
\hline
1 & 1 & a & b \\
a & a & b & 1 \\
b & b & 1 & a \\
\end{tabular}
\end{minipage}
\hspace{0.5cm}
\begin{minipage}[b]{0.45\linewidth}
\centering
\begin{tabular}{c|ccc}
$+$ & 0 & 1 & 2 \\
\hline
0 & 0 & 1 & 2 \\
1 & 1 & 2 & 0 \\
2 & 2 & 0 & 1 \\
\end{tabular}
\end{minipage}
\end{table}

Tako prva kot tudi druga tabela predstavljata grupo $\mathbb{Z}_3$.

\newpage

\item $|G|=4$ Primer: $\exists a \in G:~a^2\neq 1$

\begin{table}[hpb]
\begin{minipage}[b]{0.45\linewidth}
\centering
\begin{tabular}{c|cccc}
$\cdot$ & 1 & a & b & c \\
\hline
1 & 1 & a & b & c \\
a & a & b & c & 1 \\
b & b & c & 1 & a \\
c & c & 1 & a & b \\
\end{tabular}
\end{minipage}
\hspace{0.5cm}
\begin{minipage}[b]{0.45\linewidth}
\centering
\begin{tabular}{c|cccc}
$+$ & 0 & 1 & 2 & 3 \\
\hline
0 & 0 & 1 & 2 & 3 \\
1 & 1 & 2 & 3 & 0 \\
2 & 2 & 3 & 0 & 1 \\
3 & 3 & 0 & 1 & 2 \\
\end{tabular}
\end{minipage}
\end{table}

Obe tabeli sta $\mathbb{Z}_4$. \\

\item Primer: $\forall x \in G: x^2=1$
\begin{table}[!hpb]
\begin{minipage}[b]{0.45\linewidth}
\centering
\begin{tabular}{c|cccc}
$\cdot$ & 1 & a & b & c \\
\hline
1 & 1 & a & b & c \\
a & a & 1 & c & b \\
b & b & c & 1 & a \\
c & c & b & a & 1 \\
\end{tabular}
\end{minipage}
\hspace{0.5cm}
\begin{minipage}[b]{0.45\linewidth}
\centering
\begin{tabular}{c|cccc}
$+$ & (0,0) & (1,0) & (0,1) & (1,1) \\
\hline
(0,0) & (0,0) & (1,0) & (0,1) & (1,1) \\
(1,0) & (1,0) & (0,0) & (1,1) & (0,1) \\
(0,1) & (0,1) & (1,1) & (0,0) & (1,0) \\
(0,0) & (1,1) & (0,1) & (1,0) & (0,0) \\
\end{tabular}
\end{minipage}
\end{table}

Če pogledamo prvo tabelo, vidimo, da je to grupa $\mathbb{Z}_2 \times \mathbb{Z}_2$. Seštevamo po komponentah, vsako komponento po modulu 2.

 Poglejmo še drugo tabelo. To je res grupa, saj v splošnem velja: Če sta $(G,\circ)$ in $(H,*)$ grupi, potem je $G\times H$ grupa za operacijo $(a,b)(c,d)=(a\circ c, b * d)$. \linebreak Enota: $(1_G,1_H)$, inverz od $(a,b)$ je $(a^{-1},b^{-1})$.\\

\item $|G|=5$ \\

Recimo najprej, da je $x^2=1$ za vsak $x \in G$. To pomeni: $\Rightarrow \exists a \in G: a^2\neq 1$. BSŠ (brez škode za splošnost): $a^2=b$.

\begin{table}[!hpb]
\begin{minipage}[b]{0.30\linewidth}
\centering
\begin{tabular}{c|ccccc}
$\cdot$ & 1 & a & b & c & d \\
\hline
1 & 1 & a & b & c & d \\
a & a & 1 & c & d & b \\
b & b & d & 1 & a & c \\
c & c & b & d & 1 & a \\
d & d & c & a & b & 1 \\
\end{tabular}
\end{minipage}
\hspace{0.5cm}
\begin{minipage}[b]{0.30\linewidth}
\centering
\begin{tabular}{c|ccccc}
$\cdot$ & 1 & a & b & c & d \\
\hline
1 & 1 & a & b & c & d \\
a & a & b & c & d & 1 \\
b & b & c & d & 1 & a \\
c & c & d & 1 & a & b \\
d & d & 1 & a & b & c \\
\end{tabular}
\end{minipage}
\hspace{0.5cm}
\begin{minipage}[b]{0.30\linewidth}
\centering
\begin{tabular}{c|ccccc}
$+$ & 0 & 1 & 2 & 3 & 4 \\
\hline
0 & 0 & 1 & 2 & 3 & 4 \\
1 & 1 & 2 & 3 & 4 & 0 \\
2 & 2 & 3 & 4 & 0 & 1 \\
3 & 3 & 4 & 0 & 1 & 2 \\
4 & 4 & 0 & 1 & 2 & 3 \\
\end{tabular}
\end{minipage}
\end{table}

$\mathbb{Z}_n$ je vedno komutativna grupa. Kartezični produkt komutativnih grup je komutativna grupa, če operacijo definiramo po komponentah. Iz tega sledi, da so vse grupe moči največ 5 komutativne. \\

Če pogledamo prvo tabelo, vidimo, da to ni grupa. Razlog, da to ni grupa: ker $(ab)d=cd=a \neq d=ac=a(bd)$. Preostali dve tabeli pa sta grupi $\mathbb{Z}_5$.

\end{itemize}

\newpage 

\begin{center}
\subsubsection{Grupe permutacij}
\end{center}

\noindent Vemo že, da je množica $S(A)$ vseh bijektivnih preslikav $A\rightarrow A$ grupa za kompozitum. Če je $A$ končna množica, bijektivno preslikavo $A \rightarrow A$ imenujemo \textbf{permutacija} množice $A$. Ko gledamo permutacije, elemente množice $A$ lahko preimenujemo in se lastnosti permutacij ne spremenijo. Zato običajno vzamemo $A=\{ 1,\ldots,n \}$. \\

\noindent Grupa $(S(\{ 1,\ldots,n\}),\cdot)$ vseh permutacij množice $\{1,\ldots,n \}$ se imenuje \textbf{simetrična grupa reda $n$}. Oznaka: $S_n$. $|S_n|=n!$. Če je $n \geq 3$, $S_n$ ni komutativna grupa. \\

\noindent Permutacijo $\pi: \{1,\ldots,n \} \rightarrow \{1,\ldots,n\}$ običajno zapišemo v obliki: 
$\pi = \bigl(
\begin{smallmatrix}
1 & 2 & \cdots & n \\
\pi(1) & \pi(2) & \cdots & \pi(n)
\end{smallmatrix}\bigr) $ \\

\noindent PRIMER: 
\begin{itemize}
\item $S_3$ vsebuje permutacije 
$id=\bigl(
\begin{smallmatrix}
1 & 2 & 3 \\
1 & 2 & 3
\end{smallmatrix}\bigr), 
~ a=\bigl(
\begin{smallmatrix}
1 & 2 & 3 \\
1 & 3 & 2
\end{smallmatrix}\bigr),
~ b=\bigl(
\begin{smallmatrix}
1 & 2 & 3 \\
2 & 1 & 3
\end{smallmatrix}\bigr),
~ c=\bigl(
\begin{smallmatrix}
1 & 2 & 3 \\
2 & 3 & 1
\end{smallmatrix}\bigr), \linebreak
~ d=\bigl(
\begin{smallmatrix}
1 & 2 & 3 \\
3 & 1 & 2
\end{smallmatrix}\bigr),
~ f=\bigl(
\begin{smallmatrix}
1 & 2 & 3 \\
3 & 2 & 1
\end{smallmatrix}\bigr) $ \\

Operacija na $S_n$ je običajen kompozitum preslikav $(\pi_1 \circ \pi_2)(x)=\pi_1(\pi_2(x)), \linebreak \forall x \in \{ 1,\ldots,n \}$. \\
\end{itemize}

\noindent PRIMER: 
\begin{itemize}
\item
$a(b(1))=a(2)=3$, \\
$a(b(2))=a(1)=1$, \\
$a(b(3))=a(3)=2$, \\
$\Rightarrow a \circ b = \bigl(\begin{smallmatrix} 1 & 2 & 3 \\ 3 & 1 & 2 \end{smallmatrix}\bigr) = d$ .

\begin{table}[hpb]
\centering
\label{tab:krena}
\caption{Tabela za množenje za $S_3$}
\begin{tabular}{c|c|c|c|c|c|c}
$\circ$ & id & a & b & c & d & f \\ 
\hline
id & id & a & b & c & d & f \\
\hline
a & a & id & d & f & b & c \\
\hline
b & b & c & id & a & f & d \\
\hline
c & c & b & f & d & id & a \\
\hline
d & d & f & a & id & c & b \\
\hline
f & f & d & c & b & a & id \\
\end{tabular}
\end{table}
\begin{align*}
(a\circ c)(1)& =a(c(1))=a(2)=3 \\
(a\circ c)(2)& =a(c(2))=a(3)=2  \\
(a\circ c)(3)& =1  \\
(c\circ a)(1)& =c(c(1))=c(2)=3  \\
\quad
(c\circ c)(2)&=c(c(2))=c(3)=1  \\
(c\circ c)(3)&=2  \\
(c\circ a)(1)&=c(1)=2  \\
(c\circ a)(2)&=c(3)=1  \\
\quad
(c\circ a)(3)&=c(2)=3  \\
d^2&=(c^2)^2=c^4=c 
\end{align*}
 
Tabela za množenje na $S_3$ ni simetrična glede na diagonalo, ker $S_3$ ni komutativn grupa. \\
\end{itemize}

\begin{definicija}
Naj bodo $a_1,\ldots,a_k\in \{1,\ldots,n\}$ paroma različni.  Permutacija $\sigma \in S_n$, za katero velja $\sigma(a_i)=a_{i+1}$ za $i=1,\ldots,k-1$, $\sigma(a_n)=a_1$ in $\sigma(a)=a$ za vse \linebreak $a \in \{1,\ldots,n\}\backslash \{a_1,\ldots,a_n\}$, se imenuje \textbf{cikel dolžine $k$}. Oznaka: $\sigma=(a_1,a_2,\ldots,a_k)$. \linebreak Cikla $(a_1,\ldots,a_k)$ in $(b_1,\ldots,b_k)$ sta \textbf{disjunktivna}, kadar sta  množici $\{a_1,\ldots,a_k\}$ in $\{b_1,\ldots,b_k\}$ disjunktni.  Disjunktna cikla vedno komutirata.\\
\end{definicija}

\noindent Edini cikel dolžine 1 je identiteta. Cikel dolžine 2 se imenuje \textbf{transpozicija}. \\

\noindent PRIMER:
\begin{itemize}
\item $a=(2~3),~b=(1~2),~f=(1~3)$ so transpozicije v $S_3$, $c=(1~2~3)$ in $d=(1~3~2)$ pa sta 3-cikla v $S_3$.\\
\end{itemize}

\begin{izrek}
Vsako permutacijo lahko zapišemo kot kompozitum disjunktnih ciklov. Ti cikli med seboj komutirajo in so do vrstnega reda s permutacijo enolično določeni.\\
\end{izrek}

\noindent PRIMER:
\begin{itemize}
\item 
$
\bigl(
\begin{smallmatrix}
1 & 2 & 3 & 4 & 5 & 6 & 7 \\
5 & 2 & 1 & 3 & 4 & 7 & 6
\end{smallmatrix}
\bigr)
= 
\bigl(
\begin{smallmatrix}
1,&5,&4,&3\bigr)\bigl(2\bigr)\bigl(6,&7
\end{smallmatrix}\bigr)$ \\

Cikle dolžine 1 v produktu (kompozitumu) običajno spuščamo, ker so enaki identiteti.\\
\end{itemize}

\newpage 

\begin{proof}
Disjunktni cikli vedno komutirajo, zato moramo dokazati le obstoj in enoličnost (do vrstnega reda natančno) razcepa permutacije na produkt (kompozitum) disjunktnih ciklov.\\

\textit{Obstoj}: Obstoj bomo dokazali z indukcijo na $n$ ($n$ je moč množice, na kateri delujejo permutacije). 

\begin{gather*}
n=1: S_1=\{ id\} \\
n=2: S_2=\{ id, (1~2)\} \\
\text{Za $n \leq 2$ izrek očitno velja}. \\
\end{gather*}

\noindent Naj bo $n \geq 3$ in predpostavimo, da izrek velja za vse permutacijske grupe $S_m$, kjer je $m < n$. \\

\noindent $\pi \in S_n$ naj bo poljubna premutacija. \\

Oglejmo si množico $\{ 1,\pi(1),\pi^2(1),\pi^3(1),\ldots\}$ \\

$(\pi^2=\pi \circ \pi,~\pi^3=\pi \circ \pi \circ \pi,\ldots)$\\

\noindent Ta množica je končna, saj je podmnožica množice $\{1,2,\ldots,n\}$. \\

Zato obstajata $k$ in $l$, $0 \leq l < k$, da je $\pi^k(1)=\pi^l(1)$.\\

\noindent Naj bo $k$ najmanjši, za katerega tak $l$ obstaja: $\pi^0(1)=1 = id(1)$. \\

\noindent Dokažimo, da je $l=0$. \\

\noindent Recimo, da je $l \geq 1$. \\

$\pi^2(l)=\pi^k(1)$ lahko na levi komponiramo s $\pi^{-1}$, ker je $\pi$ bijekcija $\Rightarrow \pi^{l-1}(1)=\pi^{k-1}(1), \linebreak 0\leq l-1 < k-1$. \\

To je v protislovju z minimalnostjo $k$-ja. Torej je $l=0$ in $\pi^k(1)=1$. \\

Označimo $a_1=1,~a_2=\pi(1),\ldots,a_k=\pi^{k-1}(1)$. \\

Števila $a_1,\ldots,a_k$ so paroma različna in velja $\pi(a_i)=a_{i+1}$, za $i=1,\ldots,k-1$ in $\pi(a_k)=a_1$. \\

Cikel $(a_1,\ldots,a_k)$ na množici $\{a_1,\ldots,a_k\}$ deluje enako kot permutacija $\pi$. \\

\noindent Definirajmo $\rho=(a_1,a_2,\ldots,a_k)^{-1} \circ \pi$. \\

Potem je $\rho(a_1)=a_1$ za vsak  $i=1,\ldots,k$. $\rho$ lahko torej gledamo kot permutacijo množice $\{1,\ldots,n\} \backslash \{a_1,\ldots,a_k\}$. \\

Ta množica ima največ $n-1$ elementov, zato po indukcijski predpostavki $\rho$ lahko zapišemo kot produkt dijskunktnih ciklov, ki ne vsebujejo elementov $a_1,\ldots,a_k$. \\

\noindent Potem pa tudi $\pi=(a_1,\ldots,a_n)=\rho$ lahko zapišemo kot produkt  disjunktnih ciklov. \\

\noindent \textit{Enoličnost}: Recimo, da je $\sigma_1 \sigma_2 \ldots \sigma_k= \rho_1 \rho_2 \ldots \rho_l = \pi$, kjer so $\sigma$ disjunktni cikli in $\rho_j$ disjunktni cikli. \\

Predpostavimo lahko, da je najmanjši element cikla vedno napisan na začetku. \\

Ker disjunktni cikli komutirajo, lahko predpostavimo tudi, da je najmanjši element $\sigma_i$ manjši od najmanjšega elementa v $\sigma_j$, če je $i<j$, in isto velja za $\rho_i$ in $\rho_j$. 

\begin{align*}
\Rightarrow \sigma_1 & =(1,a_1,\ldots,a_r),~ \rho_1=(1,b_2,\ldots,b_s) \\
a_2 & =\pi(1),~a_3=\pi(a_2)=\pi^2(1),\ldots,a_r=\pi^{r-1}(1), \\
a_2 & =\pi(1),~b_3=(\pi^2(1),\ldots,b_s=\pi^{s-1}(1) \\
\Rightarrow r & =s~in~a=b~za~i=2,\ldots,r-1 \Rightarrow \sigma_1=\rho_1 \\
\end{align*}

\noindent Na nek način (oz. z indukcijo) dokažimo še $\sigma_2=\rho_2,\ldots$ (in v posebnem primeru $l=k$).\\
\end{proof}

\begin{trditev}
Vsak cikel je produkt transpozicij.\\
\end{trditev}

\begin{proof}
$(a_1,\ldots,a_k)=(a_1,a_k)(a_1,a_{k-1})\ldots(a_1,a_2)$. \\
\end{proof}

\begin{posl}
Vsaka permutacija je produkt transpozicij.\\
\end{posl}

\noindent Zapis permutacije kot produkt transpozicij ni enoličen. Na primer, $id=(1~2)(1~2)$.\\

\begin{definicija}
Za identiteto definiramo $s(id)=1$. Če je $\sigma$ cikel dolžine $k$, definiramo \linebreak $s(\sigma)=(-1)^{k+1}$. Če je $\pi \in S_n$ poljubna permutacija, jo lahko zapišemo kot produkt \linebreak disjunktnih ciklov $\pi=\sigma_1 \sigma_2 \ldots \sigma_m$. \\
\end{definicija}

\newpage 

\noindent Definiramo $s(\pi)=s(\sigma_1)s(\sigma_2)\ldots s(\sigma_m)$. \\

\noindent Ker je zapis permutacije kot produkt disjunktnih ciklov do vrstnega reda enoličen, je \linebreak definicija števila $s(\pi)$ dobra. $S$ je preslikava iz $S_n$ v $\{-1,1\}$.

\begin{align*}
s:~S_n & \rightarrow \{-1,1\} \\
\pi & \mapsto s(\pi) \\
\end{align*}

je dobro definirana preslikava. \\

\noindent Številu $s(\pi)$ pravimo \textbf{znak permutacije} $\pi$. Namesto $s(\pi)$ se včasih piše tudi $sgn(\pi)$.
\\

\noindent PRIMER:
\begin{itemize}
\item
\begin{align*}
s((1~3)(2~6~9~7)(4~5~8)) & = s((1~3)~s((2~6~9~7))~s((4~5~8))  \\ 
& = -1\cdot (-1)\cdot 1  \\
& = 1 
\end{align*}
\end{itemize}
Znak vsake transpozicije je $-1$. \\

\begin{trditev}
Če je $\tau \in S_n$ poljubna transpozicija in $\pi \in S_n$ poljubna permutacija, potem je $s(\tau \pi)=-s(\pi)~(=s(\tau)s(\pi))$.\\
\end{trditev}

\begin{proof} ~\\

\underline{1. možnost}: Naj bo $\pi=\sigma_1\ldots \sigma_m$ zapis permutacije $\pi$ kot produkt disjunktnih ciklov. \\

Znak permutacije $\sigma_1\ldots \sigma_m$ se seveda ne spremeni, če v ta produkt dodamo cikle dolžine 1. Zato lahko predpostavimo, da se vsako od števil pojavi (natanko enkrat) v kakšnem od ciklov $\sigma$. \\

Števili iz transpozicije $\tau$ sta lahko v enem ali dveh izmed ciklov $\sigma$. \\

\noindent ? števili iz transpozicije $\tau$ sta v dveh ciklih. Lahko predpostavimo, da je $\tau=(a_1,b_1), \linebreak \sigma_n=(a_1,\ldots,a_k)$ in $\sigma_2=(b_1,\ldots,b_l)$. \\

Potem je 
\begin{align*}
s(\pi) & = s(\sigma_1)s(\sigma_2)\ldots s(\sigma_m)\\
& = (-1)^{k+1}\ldots(-1)^{l+1?}~s(\sigma_3)\ldots s(\sigma_m)\\
& =(-1)^{?} s(\sigma_3)\ldots s(\sigma_m) \\
\tau \pi & =(a_1,a_2,\ldots,a_?,b_1,b_2,\ldots,b_?)\sigma_3\ldots \sigma_n
\end{align*}

\noindent To je zapis $\tau \pi$ na produkt disjunktnih ciklov, zato je
$s(\tau \pi)=(-1)^{k+l+1}s(\sigma_3)\ldots s(\sigma_m)= \linebreak = -s(\pi)$
\\

\noindent \underline{2. možnost}: Števili iz transpozicije $\tau$ sta v istem ciklu. \\

Predpostavimo lahko, da je $\tau=(a_1,a_k=$ in $\sigma_1=(a_1,\ldots,a_k,\ldots,a_l)$. \\

Potem je 
\begin{align*}
s(\pi) & = s(\sigma_1)\ldots s(\sigma_m) \\
& = (-1)^{l+1}s(\sigma_2)\ldots s(\sigma_m) \\
\tau \pi & = (a_1,a_2,\ldots,a_{k-1})(a_k,a_{k+1},\ldots,a_l)\sigma_2\ldots \sigma_m 
\end{align*} 

To je zapis $\tau  \pi$ kot produkt disjunktnih ciklov, zato je 
\begin{align*}
s(\tau \pi) & =(-1)^k(-1)^{?}s(\sigma_2)\ldots s(\sigma_m)\\
& =(-1)^{?} s(\sigma_2)\ldots s(\sigma_m)\\
& =-s(\pi)
\end{align*}

\end{proof}

\begin{posl}
Če je $\pi=\tau_1 \ldots \tau_n$, kjer so $\tau_i$ transpozicije, potem je 
$$ s(\pi)=(-1)^m = 
\begin{cases}
1; & m~sodo \\
-1; & m~liho \\
\end{cases} 
$$
\end{posl}

\begin{posl}
Če sta $\pi_1$ in $\pi_2$ poljubni permutaciji, je $s(\pi_1\pi_2)=s(\pi_1)s(\pi_2)$\\
\end{posl}

\begin{proof}
Naj bo $\pi_1=\tau_1\ldots \tau_k$ in $\pi_2=\rho_1\ldots \rho_l$, kjer so $\tau_i$ in $\rho_j$ transpozicije. \\

Potem je 
\begin{align*}
s(\pi_1\pi_2)&= s(\tau_1\ldots\tau_k \rho_1 \ldots \rho_l)  \\
& = (-1)^{k+l}  \\
& = (-1)^k(-1)^l  \\
& = s(\pi_1)s(\pi_2)
\end{align*}

\end{proof}

\begin{posl}
Iste permutacije ne moremo zapisati kot produkt sodega števila transpozicij in kot produkt lihega števila transpozicij.\\
\end{posl}

\begin{proof} (\textsc{posledice})
Naj bo $\pi=\tau_1\ldots \tau_k=\rho_1\ldots \rho_l$, kjer so $\tau_i$ in $\rho_j$ transpozicije. Znak permutacije $\pi$ je po definiciji neodvisen od teh dveh zapisov, odvisen je le od permutacije. Po predprejšnji posledici je  $s(\pi)=(-1)^k=(-1)^l\Rightarrow k~ \text{in}~l~\text{sta obe sodi ali obe lihi}$\\
\end{proof}

\begin{definicija}
Permutaciji, ki jo napišemo kot produkt sodega števila transpozicij, pravimo \textbf{soda permutacija}, permutaciji, ki jo lahko zapišemo kot produkt liheha števila transpozicij, pa pravimo \textbf{liha permutacija}. \\
\end{definicija}

\begin{trditev}
Produkt sodih permutacij je soda permutacija, inverz sode permutacije je soda permutacija.\\
\end{trditev}

\begin{proof}
Za produkt to že vemo, saj je $s(\pi_1\pi_2)=s(\pi_1)s(\pi_2)$. \\

\noindent Inverz: Naj bo $\pi=\tau_1\ldots \tau_m$, kjer so $\tau_i$ transpozicije. \\

Potem je $s(\pi)=(-1)^m$ in $\pi^{-1}=\tau_m^{-1}\ldots\tau_1^1=\tau_m\ldots \tau_1$. \\

Torej je $s(\pi^{-1})=(-1)^m=s(\pi)$. \\

\noindent Dokazali smo tudi, da je inverz lihe permutacije liha permutacija.\\
\end{proof}


\noindent Naj $A_n$ označuje množico sodih permutacij v $S_n$. Kompozitum sodih permutacij je soda permutacija (po prejšnji trditvi), torej je kompozitum notranja binarna operacija na množici $A_n$. Ta operacija je seveda asociativna. Identiteta, ki je enota za operacijo $\circ$, leži v $A_n$. Po prejšnji trditvi pa tudi inverz vsakega elementa iz $A_n$ leži v $A_n$. Torej je $A_n$ grupa za operacijo $\circ$. Rečemo ji \textbf{alternirajoča grupa reda $n$}. \\

\noindent Naj bo $\tau$ poljubna transpozicija. Potem je preslikava $\pi\mapsto \tau\pi$ bijekcija iz $A_n$ v množico lihih permutacij v $S_n$. \\

\noindent Injektivnost: $\tau^{-1}:~\tau\pi_1=\tau\pi_2 \Rightarrow \pi_1=\pi_2$ \\

\noindent Surjektivnost: $\rho$ liha permutacija $\Rightarrow$ def.: $\pi=\tau^{-1}\rho$ soda $\tau\pi=\tau\tau^{-1}\rho=\rho \Rightarrow \linebreak \Rightarrow A_n$ ima $\frac{n!}{2}$ elementov.

\newpage

\begin{center}
\subsection{Podgrupe}
\end{center}

\begin{definicija}
Naj bo $(G,\cdot)$ grupa. Neprazna podmnožica $H\subseteq G$ je \textbf{podgrupa} grupe $(G,\cdot)$, kadar velja:
\begin{enumerate}
\item[(1)] zaprtost za operacijo: če sta $x,y\in H$, mora biti $x,y\in H$,
\item[(2)] zaprtost za invertiranje: če je $x\in H$, mora biti $x^{-1}\in H$\\
\end{enumerate}
\end{definicija}

\begin{trditev}
Naj bo $H$ podgrupa grupe $G$ in $e$ enota grupa $G$. Potem je $e\in H$.\\
\end{trditev}

\begin{proof}
$H\neq \emptyset \Rightarrow \exists a \in H$. Po (2) je $a'\in H$. Po (1) je $e=aa^{-1}\in H$. \\
\end{proof}

\noindent Če je $H$ podgrupa grupe $(G,\cdot)$, je množenje notranja binarna operacija na $H$, ki je asociativna, saj se asociativnost prenese iz $G$. $H$ vsebuje tudi enoto in vse svoje inverze. Torej je $(H,\cdot)$ grupa.\\

\noindent PRIMER:
\begin{itemize}
\item $(\mathbb{Z},+)$ je podgrupa v $(\mathbb{Q},+)$, ta je pogrupa $(\mathbb{R},+)$, ta je podgrupa $(\mathbb{C},+)$
\item $(\mathbb{Q} \backslash \{0\},\cdot)$ je podgrupa $(\mathbb{R} \backslash \{0\},\cdot),~(0,\infty)$ je podgrupa $(\mathbb{R} \backslash \{0\},\cdot)$
\item $(A_n,\circ)$ je podgrupa $(S_n,\circ)$
\item Če je $\tau$ transpozicija, je $(\{id,\tau\},\circ)$ podgrupa grupe $(S_n,\circ)$
\item Vsaka grupa $G$, ki ima vsaj dva elementa, ima vsaj dve podgrupi: \linebreak $G$ - neprazna podgrupa in $\{e\}$ - trivialna podgrupa. \\

Lahko se zgodi, da sta to edini podgrupi grupe $G$. Za primer lahko vzamemo $(\mathbb{Z}_p,+)$, kjer je $p$ praštevilo.\\
\end{itemize}

\begin{trditev}
Neprazna podmnožica $H$ grupe $(G,\cdot)$ je podgrupa natanko takrat, ko velja \linebreak $xy^{-1}\in H,~\forall x,y\in H$.\\
\end{trditev}


\begin{proof} ~~~
\begin{itemize}
\item[($\Rightarrow$)]: $H$ je podgrupa \\

Naj bosta $x,y\in H$ poljubna. \\

Ker je $H$ podgrupa, po točki (2) in definiciji sledi $y^{-1}\in H$. \\

Po točki (1) je potem $xy^{-1}\in H$.

\item[($\Leftarrow$)]: \\

Predpostavimo, da je velja $xy^{-1}\in H$ za vse $x,y\in H$. Radi bi dokazali, da je $H$ zaprta za množenje in za invertiranje. \\

Naj bosta $x,y\in H$ poljubna. \\

Potem je $xx^{-1}\in H$. \\

Torej $H$ vsebuje enoto $e$ grupe $G$. \\

Potm pa velja tudi $x^{-1}=ex^{-1}\in H$, torej je $H$ zaprta za invertiranje. \\

Enako velja tudi $y^{-1}\in H$. \\

Potem pa je  $xy=x(y^{-1})^{-1} \in H$, torej je $H$ zaprta za množenje. \\
\end{itemize}
\end{proof}

\begin{posl}
Naj bo $(G,+)$ aditivno pisana Abelova grupa. Potem je $\emptyset \neq H \subseteq G$ podgrupa grupe $G$ natanko takrat, ko za vsaka $x,y\in H$ velja $x-y\in H$ (zaprtost za odštevanje).\\
\end{posl}

\begin{trditev}
Presek poljubne družine podgrupa je podgrupa.\\
\end{trditev}

\begin{proof}
Naj bo $\{H_j\}_{j\in J}$ družina podgrup grupe $G$ ($J$ je poljubna indeksna množica) in naj bo $H=\bigcap_{j\in J} H_j$. \\

Naj bosta $x,y\in H$ poljubna. \\

Potem sta $x,y\in H_j$ za vsak $j\in J$. \\

Po trditvi je zato $xy^{-1}\in H_j$ za vsak $j\in J$. \\

Torej je $xy^{-1}\in \bigcap_{j\in J} H_j = H$.\\
\end{proof}

\noindent Unija podgrup v splošnem ni podgrupa. Za primer lahko vzamemo:
\begin{itemize}
\item $G=S_3$\\
$H_1=\{id,(1,2)\}$ in $H_2=\{id,(1,3)\}$ sta podgrupi v $S_3$. \\$H_1 \cup H_2$ ni podgrupa, saj $(1,2)(1,3)\notin H_1 \cup H_2$
\end{itemize}
\newpage 
\subsection{Homomorfizem grup}

\begin{definicija}
Naj bosta $(G,\circ)$ in $(H,*)$ grupi. Preslikava $f:G\rightarrow H$ je \textbf{homomorfizem grup}, kadar velja $f(x\circ y)=f(x)*f(y)$ za vsaka $x,y\in G$.\\
\end{definicija}

\noindent PRIMERI:

\begin{itemize}

\item Naj bo $n\in \mathbb{Z}$ in definiramo preslikavo 
\begin{align*}
f:(\mathbb{Z},+) & \rightarrow (\mathbb{Z},+)\\
x& \mapsto nx
\end{align*}
\begin{align*}
f(x+y)& =n(x+y)  \\
f(x)+f(y)& =nx+ny 
\end{align*} 
\begin{center}
$\Rightarrow f~\text{je homomorfizem grup}$
\end{center}

\item 
\begin{align*}
f:(\mathbb{Z},+) & \rightarrow (\mathbb{Q}\backslash \{0\}\cdot)\\
x & \mapsto 2^x
\end{align*} 
\begin{center}
$f(x+y)=x^{x+y}=2^x\cdot 2^y=f(x)\cdot f(y) \Rightarrow f~\text{je homomorfizem grup}$
\end{center}

\item 
\begin{align*}
f:(\mathbb{R}\backslash \{0\},\cdot)& \rightarrow((0,\infty),\cdot)\\
x & \mapsto f|x|
\end{align*} \\
To je tudi homomorfizem grup, saj je $f(xy)=|xy|=|x|\cdot |y|=f(x)\cdot f(y)$

\item 
\begin{align*}
s: S_n & \rightarrow \{-1,1\}\\
\pi & \mapsto s(\pi)
\end{align*}

Je homomorfizem grup, saj vemo, da je $s(\pi_1\circ \pi_2)=s(\pi_1)s(\pi_2)$

\item $id:G\rightarrow G$ je vedno homomorfizem
\item 
\begin{align*}
f:G & \rightarrow H \\
x & \mapsto e,~\forall x\in G
\end{align*}
(kjer je $e$ enota grupe $H$) je tudi homomorfizem grup
\item 
\begin{align*}
f:(\mathbb{R}^2,+)&\rightarrow(\mathbb{R},+) \\ 
(x,y) & \mapsto ax+by
\end{align*}
kjer sta $a,b$ poljubni konstanti 
\begin{align*}
f((x,y)+(z,w))& =f(x+z,y+w)\\
& =a(x+z)+b(y+w) \\
& =ax+az+by+bw \\
& =f((x,y))+f((z,w))\\
& \Rightarrow f ~ \text{je homomorfizem grup}\\
\end{align*}

\end{itemize}

\begin{trditev}
Naj bo $f:G\rightarrow H$ homomorfizem grupi, kjer je $e$ enota v $G$ in $e'$ enota v $H$. Potem je $f(e)=e'$ in $f(a^{-1})=f(a)^{-1}$ za vsak $a \in G$. Pri tem $a^{-1}$ predstavlja inverz v $G$, $a$ pa inverz v $H$.\\
\end{trditev}

\begin{proof}
~
\begin{gather*}
f(e)=f(e\cdot e)=f(e)\cdot f(e) \\
e'=f(e)\cdot f(e)^{-1} =f(e)\\
f(e)=e' \\
f(a\cdot a^{-1})=f(a)\cdot f(a^{-1}) \\
\text{Z leve smo pomnožili z}~f(a)^{-1} \Rightarrow f(a)^{-1}=f(a^{-1})
\end{gather*}
\end{proof}

\noindent \textbf{Izomorfizem} je bijektiven homomorfizem grup.\\
\textbf{Monomorfizem} je injektiven homomorfizem grup.\\
\textbf{Epimorfizem} je surjektiven homomorfizem grup.\\
\textbf{Endomorfizem} je homomorfizem iz grupe vase.\\
\textbf{Avtomorfizem} je bijektiven endomorfizem.\\

\noindent $f:G\rightarrow H$ je monomorfizem $\Leftrightarrow$ za vsaka homomorfizma grupa $g,h:K\rightarrow G$ iz $f\circ g=f\circ h$ sledi $g=h$. \\

\noindent Če je $H$ Abelova, je $f:G\rightarrow H$ epimorfizem $\Leftrightarrow$ za vsaka homomorfizma grup $g,h:H\to K$ iz $g\circ f=h\circ f$ sledi $g=h$.\\

\noindent Če $H$ ni Abelova, to ni res.\\

\newpage 

\noindent PRIMER:
\begin{center}
$f:S_3\to S_3$ \\
~\\
$f(\pi)=
\begin{cases}
(1,2),& \pi~\text{liha} \\
id,& \pi~\text{soda}\\
\end{cases}$
\end{center}

\begin{trditev}
~
\begin{itemize}
\item[(a)] Kompozitum homomorfizmov grup je homomorfizem grup.
\item[(b)] Inverz izomorfizma grup je homomorfizem (in zato izomorfizem) grup.\\
\end{itemize}
\end{trditev}

\begin{proof}
~
\begin{itemize}
\item[(a)] $f:(G,\cdot)\to (H,\circ)$ in $g:(H,\circ)\to(K,*)$ naj bosta homomorfizma grup in naj bosta $x,y\in G$ poljubna. \\

Potem je
\begin{align*}
(g\circ f)(xy) & = g(f(xy)) \\
& = g(f(x)\circ f(y)) \\
& = g(f(x))*g(f(y)) \\
& \Rightarrow g \circ f: (G,\cdot)\to(K,*)\\
& \text{je homomorfizem grup}
\end{align*}

\item[(b)] Naj bo $f:G\to H$ izomorfizem. \\

Hočemo dokazati, da je $f^{-1}:H\to G$ homomorfizem grup.\\ 

Naj bosta $x,y\in H$ poljubna. \\

Ker je $f$ izomorfizem, obstajata enolična elementa $a,b\in G$, da je $x=f(a)$ in $y=f(b)$. 

\begin{align*}
f ~ \text{je homomorfizem} & \Rightarrow \\
f(ab)& =f(a)\cdot f(b) \\
f(ab)&=xy \\
f^{-1}f^{-1}(y)& =ab=f^{-1}(xy) \\
& \Rightarrow f^{-1}~ \text{je homomorfizem grup}.\\
\end{align*}
\end{itemize}
\end{proof}

\begin{definicija}
Grupi $G$ in $H$ sta \textbf{izomorfni}, če med njima obstaja \linebreak izomorfizem. Oznaka: $G \cong H$.\\
\end{definicija}

\noindent V algebri med izomorfiznimi grupami običajno ne ločujemo. Npr. $\mathbb{Z}_2$  označujemo vsako grupo z 2 elementoma. Npr., če je $\tau$ transpozicija, je $\{id_?\tau\}=\mathbb{Z}_2$.\\

\begin{definicija}
Naj bo $f:G\to H$ homomorfizem grup. Zalogi vrednosti tega  homomorfizma pravimo \textbf{slika} homomorfizma $f$ in jo označimo \emph{im $f$}.  Množico $\{x\in G,f(x)=e\}$ (kjer je $e$ enota grupe $H$) pa imenujemo \textbf{jedro} preslikave $f$ in ga označimo \emph{ker $f$}.\\
\end{definicija}


\begin{trditev}
Naj bo $f:G\to H$ homomorfizem grup. Potem je \emph{ker$f$} podgrupa grupe $G$ in \emph{im$f$} podgrupa grupe $H$.\\
\end{trditev}

\begin{proof} ~\\

\noindent Ker je $f(e_G)=e_H$, je množica ker$f$ neprazna. \\

Naj bosta $a,b\in \text{ker}f$. \\

Potem je $f(a)=f(b)=e_H$.

\begin{equation*}
f(ab')=f(a)f(b^{-1})=f(a)f(b)^{-1}=e_H e_H^{-1} = e_H \Rightarrow ab^{-1}\in \text{ker}f \\
\end{equation*}

Torej je ker$f$ podgrupa grupe $G$. \\

Zaloga vrednosti je vedno neprazna. \\

\noindent Naj bosta $a,b\in \text{im}f$. \\

To pomeni, da obstajata $x,y\in G$, da je $a=f(x)$ in $b=f(y)$. \\

Potem je $ab^{-1}=f(x)f(y)^{-1}=f(xy^{-1})$. \\

Torej je $ab^{-1}\in \text{im}f$ in slika homomorfizma $f$ je podgrupa grupe $H$.\\
\end{proof}

\newpage 

\noindent Očitno velja $f$ je surjektiven homomorfizem $f:G\to H \Leftrightarrow \text{im}f=H$.\\

\begin{izrek}
Homomorfizem grup $f:G \to H$ je injektiven $\Leftrightarrow \text{ker}f=\{ 1 \} $. Pri tem je 1 enota grupe $G$. \\
\end{izrek}

\begin{proof}
~\\
\begin{itemize}
\item[($\Rightarrow$)] ~ \\

Vedno je $f(1_G)=1_H.~ $ker$f=\{x\in G,f(x)=1_H\}$ \\

\noindent Predpostavimo, da je $f$ injektiven homomorfizem in naj bo $x\in \text{ker}f$ poljuben. 
\begin{center}
$f(x)=1_H=f(1_G)$ 
\end{center}
Ker je $f$ injektiven, od tod sledi $x=1_G$. Torej je ker$f=\{1_G\}$ \\

\item[($\Leftarrow$)] ~ \\

Predpostavimo, da je ker$f=\{1_G\}$. \\

Hočemo dokazati, da je $f$ injektivna. \\

\noindent Naj bosta $x,y\in G$ poljubna, za katera je $f(x)=f(y)$. 
\begin{gather*}
f(x)=f(y)~/\cdot f(y)^{-1} \\
f(xy^{-1})=f(x)f(y)^{-1}=1_H
\end{gather*}

\noindent (pri znaku = vemo, da to velja, ker je f homomorfizem) \\

\noindent Ker je jedro preslikavde $f$ po predpostavki trivialno, je $xy^{-1}=1_G$ oziroma ko enačbo pomnožimo z $y$ dobimo, da je $x=y$ in posledično dokažemo, da je $f$ torej injektivna.\\
\end{itemize}
\end{proof}

\newpage 

\noindent PRIMER: 
\begin{itemize}
\item 
\begin{align*}
f(\mathbb{R},+)& \to((0,\infty),\cdot)\\
x & \mapsto 2^x
\end{align*}
Vemo, da je to homomorfizem grup. Ali je injektiven? \\

Enota v grupi $((0,\infty),\cdot)$ je 1. \\

\noindent Kdaj je $x\in \text{ker}f$? 
\begin{equation*}
\Rightarrow f(x)=1 \Leftrightarrow 2^x=1
\end{equation*}

To je res le v primeru, ko je $x=0$ enota grupe $(\mathbb{R},+)$. \\

Torej je ker$f=\{0\}$ in zato je $f$ injektiven homomorfizem.\\

Kdaj je $z\in\text{ker}f$? 
\begin{equation*}
\Leftrightarrow f(z)=1\Leftrightarrow |z|=1 \Leftrightarrow z=\cos\varphi + i\sin\varphi~\text{za nek}~\varphi\in[0,2\pi].\\
\end{equation*}
\begin{equation*}
\text{ker}f=\{\cos\varphi+i\sin\varphi;\varphi\in[0,2\pi]\}
\end{equation*}


Jedro vsebuje poleg enote še druge elemente, zato $f$ ni injektivna preslikava. \\

Dokazali smo tudi, da je enotska krožnica podgrupa grupe $(\mathbb{C}\backslash\{0\},\cdot)$

\end{itemize}

\newpage

\subsection{Kolobarji}

\begin{definicija}
Neprazna množica $K$ z operacijama seštevanja in množenja je \textbf{kolobar}, kadar velja:
\begin{enumerate}
\item[1)] $(K,+)$ je Abelova grupa (in tudi pišemo jo aditivno)
\item[2)] $(K,\cdot)$ je polgrupa (to pomeni: množenje je asociativno)
\item[3)] velja \emph{distributivnost}: $a(b+c)=ab+ac$ in $(b+a)a=ba+ca,~\forall a,b,c\in K$\\
\end{enumerate}
\end{definicija}

\noindent Kolobar $K$ ima \emph{enoto} (ali \emph{enico}), če je $(K,\cdot)$ monoid. Enoto (kadar obstaja) običajno \linebreak označujemo z 1. \\ 

\noindent Kolobar je \emph{komutativen}, kadar je množenje v $K$ komutativno.\\

\begin{trditev}
V kolobarju velja:
\begin{itemize}
\item[1)] $a\cdot 0=0\cdot a=0,~\forall a\in K$
\item[2)] $a(-b)=(-a)b=-(ab)$ in $(-a)(-b)=ab,~\forall a,b\in K$\\
\end{itemize}
\end{trditev}

\begin{proof} ~
\begin{itemize}
\item[1)] $a\cdot 0=a(0+0)=a\cdot 0 + a \cdot 0$. \\

Na obeh straneh odštejemo $a\cdot 0$ in dobimo $0=a\cdot 0$.\\

Na enak način dokažemo $0\cdot a=0,~\forall a \in K$.\\

\item[2)] $a(-b)+ab=a(-b+b)=a\cdot 0=0 \Rightarrow a(-b)$ je nasproten element elementa $ab:~a(-b)=-(ab)$ \\

Enako dokažemo $(-a)\cdot b=-(ab)$.\\

Enakost $(-a)(-b)=ab$ dobimo tako, da enakost $a(-b)$ uporabimo za $-a$ in $b$.
\end{itemize}
\end{proof}

\newpage 

\noindent \textsc{Primeri kolobarjev}:
\begin{itemize}
\item Številski kolobarji: $(\mathbb{Z},+,\cdot),(\mathbb{Q},+,\cdot),(\mathbb{R},+,\cdot),(\mathbb{C},+,\cdot)$
\item Na množici $F(\mathbb{R})$ vseh funkcija $\mathbb{R}\to\mathbb{R}$ definiramo seštevanje in množenje po točkah:
\begin{gather*}
(f+g)(x)=f(x)+g(x),~\forall x \in \mathbb{R} \\
(f\cdot g)(x)=f(x)\cdot g(x),~\forall x \in \mathbb{R}
\end{gather*}
Hitro se da preveriti, da je $(F(\mathbb{R}),+,\cdot)$ kolobar. Ta kolobar ima enoto \linebreak $e:\mathbb{R} \to \mathbb{R},~e(x)=1$ za vsak $x\in \mathbb{R}$. 
\begin{equation*}
(e\cdot f)(x)=e(x)\cdot f(x)=1\cdot f(x)=f(x)~\text{za vsak}~f\in f(\mathbb{R})~\text{in za vsak}~x\in \mathbb{R}
\end{equation*}
$\Rightarrow e\cdot f=f$, enako vidimo, da je $f\cdot e=f,~\forall f\in F(\mathbb{R})$. \\

Kolobar je komutativen: $(f\cdot g)(x)=f(x)\cdot g(x)=g(x)\cdot f(x)=(gf)(x)$, za vse $ f,g$

\item Polinom z realnimi koeficienti je izraz oblike 
\begin{center}
$p(x)=a_nx^n+a_{n-1}x^{n-1}+\cdots+a_0$, kjer so $a_i\in \mathbb{R}$
\end{center}
Množico polinomov z realnimi koeficietni označujemo z $\mathbb{R}[X]$. Na $\mathbb{R}[X]$ definiramo običajno seštevanje in množenje polinomov. Za to seštevanje in množenje je $\mathbb{R}[X]$ kolobar (dokaži doma). Ta kolobar je komutativen in ima enoto $p(x)=1$.

\item $\mathbb{Z}_n$. \\

Ostanek števila $a$ pri deljenju z $n$ bomo označili z ?. Potem je \begin{equation*}
\mathbb{Z}_n=\{[0],[1],\ldots,[n-1]\}
\end{equation*}

Na $\mathbb{Z}_n$ definiramo operaciji $+$ in $\cdot$ s predpisoma $[a]+[b]=[a+b]$ in $[a]\cdot[b]=[a\cdot b]$.\\

Preveriti moramo, da je definicija dobra, torej, da sta $[a+b]$ in $[a\cdot b]$ odvisna le od ostankov števil $a$ in $b$.\\

Naj bo $[a]=[a']$ in $[b]=[b']$. To pomeni, da $n|a-a'$ in $n|b-b'$. 
\begin{equation*}
\Rightarrow n|a-a' + b-b'=a+b-(a'+b')\Rightarrow [a+b]=[a'+b']
\end{equation*}

Seštevanje je dobro definirano, saj je ostanek $[a+b]$ odvisen le od ostankov $[a]$ in $[b]$, ne pa tudi od $a$ in $b$. 
\begin{equation*}
n|a-a',~n|b-b' \Rightarrow n|(a-a')b+a'(b-b')=ab-a'b' \Rightarrow [ab]=[a'b']
\end{equation*}

Tudi množenje je dobro definirano. \\

Za tako definirano seštevanje in množenje je $\mathbb{Z}_n$ kolobar (preveri doma). Je komutativen in ima enoto $[1]$.\\

Kadar ni bojazni, da bi ostanke zamešali s celimi števili, namesto $[a]$ pišemo kar $a$. Druga oznaka: $a+n\mathbb{Z}$.

\item $(\mathbb{R}^3,+,x)$ ni kolobar, saj $x$ ni asociativen

\item Na $\mathbb{R}^3$ definiramo množenje. \\
\begin{equation*}
(x_1,y_1,z_1)\cdot (x_2,y_2,z_2)=(x_1x_2,x_1y_2+y_1z_2,z_1z_2)
\end{equation*}

Za to množenje in običajno seštevanje je $\mathbb{R}^3$ kolobar (preveri doma). \\

Kolobar ima enoto $(1,0,1)$. \\

Kolobar ni komutativen: 
\begin{equation*}
(1,0,0)\cdot(0,1,0)=(0,1,0) \neq (0,0,0) = (0,1,0)\cdot (1,0,0)
\end{equation*}

\item $A$ naj bo (aditivno pisana) Abelova grupa in End($A$) množica vseh endomorfizmov grupe $A$. \\

Dokazali smo, da je kompozitum homomorfizmov grup spet homomorfizem grup, torej je $\circ$ dobro definirana operacija na End$(A)$. Ta operacija je asociativna in ima enoto $id_A \Rightarrow (\text{End}(A),\circ)$ je monoid.\\
\end{itemize}

\noindent Dokažimo $f,g\in \text{End}(A) \Rightarrow f+g\in \text{End}(A)$ \\

\begin{proof} ~\\

$a,b\in A$
\begin{align*}
(f+g)(a+b)& =f(a+b)+g(a+b \\
& =f(a)+f(b)+g(a)+g(b)\\
& =(f+g)(a)+(f+g)(b) \\
& \Rightarrow f+g\in \text{End}(A)
\end{align*}
$\Rightarrow~ \text{seštevanje je dobro definirana operacija na End(A)}$.\\

\newpage 

\noindent Hitro se vidi, da je ta operacija asociativna in komutativna. Ta operacija ima enoto, ki je preslikava, ki vse elemente slika v 0.
\begin{center}
0: $A \rightarrow A$ \\
   $a \mapsto 0~\forall a$
\end{center}
\begin{equation*}
(f+0)(a)=f(a)+0(a)=f(a)+0=f(a),~\text{za vsak $a\in A \Rightarrow f+0=f$}
\end{equation*} 
\noindent Pri tem sta ničla pri $(f+0)(a)$ in $0(a)$ ničelni preslikavi. 0 pri $f(a) + 0$ pa je nič v $A$. \\

\noindent Inverz preslikave $f\in \text{End}(A)$ za seštevanje je preslikava - $f$, definirana s predpisom $(-f)(a)=-f(a),~\forall a\in A$. \\

\noindent Torej je $(\text{End}(A),+)$ Abelova grupa.\\

\noindent Vemo že, da je $(\text{End}(A),\circ)$ monoid.\\

\noindent Dokažimo še distributivnost. \\

Naj bodo $f,g,h\in \text{End}(A)$ poljubni endomorfizmi.\\

Radi bi dokazali, da je $f\circ(g+h)=f\circ g + f \circ h$ in $(g+h)\circ f=g\circ f + h \circ f$ \\

\noindent Naj bo $a\in A$ poljuben. \\

Potem je 
\begin{align*}
(f\circ(g+h))(a) & = f((g+h)(a)) \\
&= f(g(a)+h(a)) \\
&= f(g(a))+f(h(a)) \\
&= (f\circ g)(a)+(f\circ h)(a) \\
&= (f\circ g+f\circ h)(a) \\
&\Rightarrow f\circ (g+h)=f\circ g+f\circ h
\end{align*}
\begin{align*}
((g+h)\circ f)(a) & = (g+h)(f(a)) \\
&=g(g(a))+h(f(a)) \\
&=(g\circ f)(a)+(h\circ f)(a)\\
&=(g\circ f+h\circ f)(a)\\
&\Rightarrow (g+h)\circ f=g\circ f + h \circ f
\end{align*}
$\Rightarrow (\text{End}(A),+,\circ)$ je kolobar z enoto $id_A$. V spolšnem komutativen.\\
\end{proof}

\newpage 

\begin{definicija}
Naj bo $K$ kolobar in $a,b\in K$ taka neničelna elementa, da je $ab=0$. Elementoma $a$ in $b$ pravimo \textbf{delitelja niča}. Natančneje, $a$ je \textbf{levi delitelj niča}, $b$ pa \textbf{desni delitelj niča}.\\
\end{definicija} 

\noindent PRIMER
\begin{itemize}
\item V $\mathbb{Z}_6$ velja $[2][3]=[0]$
\item V primeru 6 je veljajo $(0,1,0)(1,0,0)=(0,0,0)$
\item $(F(\mathbb{R},+,\cdot))$ naj bo kolobar vseh realnih funkcij z operacijama po točkah. Naj bo $f(x)=
\begin{cases}
1;& x=0 \\
0;& x\neq 0
\end{cases}$ in $g(x)=
\begin{cases}
0,&x=0 \\
1,&x\neq 0
\end{cases}$. Potem je $f(x)\cdot g(x)=0,~\forall x \in \mathbb{R}$. 

\noindent Torej je $f\cdot g=0$. $f$ in $g$ sta delitelja niča.\\
\end{itemize}

\begin{definicija}
Kolobar $K$ je \textbf{obseg}, kadar ima enoto $1\neq 0$ in je vsak njegov neničelni element obrnljiv: $\forall a \in K \backslash \{0\} ~ \exists a^{-1}\in K$, da je $a\cdot a^{-1}=1=a^{-1}a$.\\
\end{definicija}

\noindent Ekvivalentno: kolobar $K$ je obseg $\Leftrightarrow K\backslash \{0\}$ grupa za množenje.\\

\noindent \underline{Opomba 1:} Če ima $K$ vsaj dva elementa, pogoj $1\neq0$ ni potreben. Če je $1=0$, namreč za vsak $x\in K$ velja $x=1\cdot x = 0\cdot x=0 \Rightarrow |K| =1$.\\

\noindent \underline{Opomba 2:} Če je $K$ nekomutativen obseg, je lahko $xy^{-1}\neq y^{-1}x$, zato izraz $\frac{x}{y}$ ni dobro definiran.\\

\begin{definicija}
\textbf{Polje} je komutativen obseg. Letos bodo vsi obsegi komutativni. Ne bomo posebej omenjali, da so komutativni.\\
\end{definicija} 

\begin{trditev}
V obsegu ni deliteljev niča.\\
\end{trditev}

\begin{proof}
Recimo, da obstajata $a,b\in K$~($K$ je obseg), da je $a\neq 0$, $b\neq 0$ in $ab=0$. \\

Ker $a\neq 0$, obstaja $a^{-1}\in K$ 
\begin{gather*}
a^{-1} ~ / ~ ab=0 \\
b=a^{-1}ab=0 \\
\text{Protislovje}
\end{gather*}
\end{proof}

\noindent PRIMERI OBSEGOV: $\mathbb{Q},\mathbb{R},\mathbb{C}$\\

\begin{trditev}
$\mathbb{Z}_n$ je obseg $\Leftrightarrow$ $n$ je praštevilo.\\
\end{trditev}

\begin{proof} ~\\

($\Rightarrow$): Predpostavimo, da je število $n$ sestavjeno: $n=p\cdot q,~1<p,q<n$. \\

\noindent Iz tega sledi: $[p][q]=[0]$. \\

\noindent $\mathbb{Z}_n$ ima torej delitelje niča, torej ni obseg.\\

($\Leftarrow$): Predpostavimo, da je $n$ praštevilo. \\

\noindent Naj bo $[a]\neq [0],[a]\in \mathbb{Z}_n$ poljuben. \\

\noindent $n$ ne deli števila $a$ \\

\noindent $a,2a,\ldots,(n-1)a$ \\

\noindent Nobeno od števil $a,2a,\ldots,(n-1)a$ ni deljiv z $n$, ker je $n$ praštevilo. \\

Velja tudi, da dajo ta števila paroma različne ostanke pri deljenju z $n$. \\

Če bi namreč veljajo $[ka]=[la]$ za neka $1\leq k < l \leq n-1$, potem bi $n$ delil $(l-k)a$, kjer je $l-k\in \{1,\ldots,n-1\}$. \\

To nas dovede do protislovja.\\

\noindent Torej so ostanki $[a],[2a],\ldots,[(n-1)a]$ paroma različni in nobeden ni $[0]$. \\

Torej je $[ka]=[1]$ za vsak $1\leq k \leq n-1$. \\

\noindent $\Rightarrow [k]=[a]^{-1}$ v $\mathbb{Z}_n \Rightarrow \mathbb{Z}_n$ je obseg.\\
\end{proof}
\newpage

\section{Končnorazsežni vektorski prostori}

\subsection{Baza in razsežnost}

\begin{definicija}
Naj bo $V$ vektorski prostor nad obsegom $O$ in $M \subseteq V$ poljubna množica. Pravimo, da je $M$ \textbf{ogrodje} prostora $V$, kadar je Lin($M$)=$V$. Pravimo tudi, da $M$ \textbf{generira} $V$ in elementom $M$ pravimo \textbf{generatorji}.\\
\end{definicija}

\noindent Ogrodje prostora $V$ je torej taka množica $M$, da vsak vektor iz $V$ lahko izrazimo kot (končno!) linearno kombinacijo elementov iz $M$. \\

\noindent PRIMERI:
\begin{itemize}
\item V $O^n$ za vsak $j=1,\ldots,n$ označimo $e_j=(0,\ldots,0,1,0,\ldots,0)$. \\

Potem je $\{e_1,\ldots,e_n\}$ ogrodje prostora $O^n$. Zakaj? \\

Naj bo $x\in O^n$ poljuben (pri čemer velja $x=x_1,x_2,\ldots,x_n$). \\

Potem je $x=x_1e_1+x_2e_2+\cdots+x_ne_n$.

\item $\mathbb{R}[X]$. \\

Ogrodje je $\{1,x,x^2,\ldots\}$. \\

Ogrodje je neskončno, a vsak polinom  $a_nx^n+a_{n-1}x^{n-1}+\cdots+a_0$ je končna linearna kombinacija polinomov $1,x,x^2,\ldots,x^n$. \\
\end{itemize}

\begin{definicija}
Vektorski prostor je \textbf{končnorazsežen}, kadar ima kakšno končno ogrodje. \\
\end{definicija}

\noindent PRIMERI:
\begin{itemize}
\item $O^n$ je končnorazsežen
\item $\mathbb{R}[X]$ ni končnorazsežen: Če je $M=\{p_1(x),\ldots,p_n(x)\}$ poljubna končna množica polinomov, potem noben polinom v Lin(M) nima stopnje večje od stopenj polinomov $p_1(x),\ldots,p_n(x)$. 
\item Prostor polinomov stopnje največ $n$ je končnorazsežen.
\item Prostor vseh funkcij $\mathbb{R}\to \mathbb{R}$ ni končnorazsežen.\\
\end{itemize}

\noindent Letos bodo vsi vektorski prostori končnorazsežni. Vsak vektorski prostor ima ogrodje, saj je Lin($V$)=$V$. Radi bi našli čim manjše ogrodje.\\

\begin{trditev}
Naj bo $M$ ogrodje vektorskega prostora $V$ in $v\in M$ tak vektor, ki \emph{pripada Lin($M\backslash \{v\})$} \footnote{$v$ lahko izrazimo kot linearno kombinacijo ostalih vektorjev iz $M$}. Potem je $M\backslash \{v\}$ tudi ogrodje prostora $V$. \\
\end{trditev}

\begin{proof}
Naj bo $x\in V$ poljuben. \\

Potem, ker je $M$ ogrodje za $V$, obstajajo $\alpha_1,\ldots,\alpha_n\in O$ in $u_1,\ldots,u_n\in M$, da je \linebreak $x=\alpha_1u_1+\cdots+\alpha_nu_n$. \\

Če je $u_i \neq v$ za vsak $i=1,\ldots,n$, je $x\in \text{Lin}(M\backslash \{v\})$. \\

\noindent Predpostavimo še, da je $v=u_i$, za nek $i$.\\

 Predpostavimo lahko, da je $v=u_1$ in $v\neq u_j$ za $j\geq 2$. \\

\noindent Ker je $v\in \text{Lin}(M\backslash \{v\})$, obstajajo $v_1,\ldots,v_m\in M\backslash \{v\}$ in $\beta_1,\ldots,b_m\in O$, da je \linebreak $v=\beta_1v_1+\ldots+\beta_mv_m$ \\

\noindent Potem je $x= \footnote{$\in \text{Lin}(M\backslash\{v\})$}  ~\alpha_1\beta_1v_1+\ldots+\alpha_1\beta_mv_m+\footnote{$\in \text{Lin}(M\backslash\{v\})$}~\alpha_2u_2+\ldots+\alpha_nu_n \in \text{Lin}(M\backslash\{v\})$. \\

\noindent Vsak vektor iz $V$ se da izraziti kot linearno kombinacijo elementov iz $M\backslash\{v\}$, torej je $M\backslash \{v\}$ ogrodje prostora $V$. \\
\end{proof}

\begin{definicija}
Vektorji $v_1,\ldots,v_n\in V$ so \textbf{linearno neodvisni}, kadar velja naslednji sklep: Če so $\alpha_1,\ldots,\alpha_n\in O$ in je $\alpha_1v_1+\ldots+\alpha_nv_n=0$, potem je $\alpha_1=\ldots=\alpha_n=0$. \\
\end{definicija}

\noindent Vektorji so \textbf{linearno odvisni}, kadar niso linearno neodvisni. \\

\noindent  Končna množica $M\subseteq V$ je \textbf{linearno neodvisna}, kadar je vsaka njena končna podmnožica sestavljena iz linearno neodvisnih vektorjev. \\

\noindent Vedno iz $\alpha_1=\ldots=\alpha_n=0$ sledi $\alpha_1v_1+\ldots+\alpha_nv_n=0$. \\

\noindent Definicija linearne neodvisnosti je obrat te implikacije. $v_1,\ldots,v_n$ so linearno neodvisni, kadar se ne more zgoditi, da je $\alpha_1v_1+\ldots+\alpha_nv_n=0$, vendar $\alpha_1,\ldots,\alpha_n$ niso vsi 0. \\

\newpage 

\noindent PRIMER:
\begin{itemize}
\item $\{0\}$ je linearno odvisna množica: $1\cdot 0=0$ ($1=\alpha\neq0$).
\item $\{e_1,\ldots,e_n\}$ je linearno neodvisna množica v $O^n$. Zakaj? \\

Recimo, da je $\alpha_1e_1+\ldots+\alpha_ne_n=(0,\ldots,0) \Rightarrow \alpha_1=\alpha_2=\ldots=\alpha_n=0$. \\

($\alpha_1e_1+\ldots+\alpha_ne_n=\alpha_1,\alpha_2,\ldots,\alpha_n$)

\item Vektorji $(1,1,-1),(1,1,2)$ in $(-2,2,2)$ so linearno odvisni, saj je \linebreak $2\cdot(1,1,-1) + 0\cdot(1,1,2)+1\cdot(-2,-2,2)=(0,0,0)$. \\
\end{itemize}

\begin{trditev}
Vektorji $v_1,\ldots,v_n\in V$ so linearno odvisni natanko takrat, ko enega od njih lahko izrazimo kot linearno kombinacijo prejšnjih. \\
\end{trditev}

\begin{proof}~

\begin{itemize}
\item[$(\Rightarrow)$]:
Naj bodo $v_1,\ldots,v_n\in V$ linearno odvisni. \\

Potem obstajajo $\alpha_1,\ldots,\alpha_n\in O$, ne vsi 0, da je $\alpha_1v_1+\ldots+\alpha_nv_n=0$. \\

Naj bo $k$ največji indeks, za katerega je $\alpha_k\neq0$. \\

Torej je $\alpha_1v_1+\ldots+\alpha_kv_k+0\cdot v_{k+1}+\ldots=0$. Zato:
\begin{gather*}
\alpha_1v_1+\ldots+\alpha_{k-1}v_{k-1}=-\alpha_kv_m / :~(-\alpha_k)^{-1} \\
v_k=-\alpha_1\alpha_k^{-1}v_1-\ldots-\alpha_{k-1}\alpha_k^{-1}v_{k-1}
\end{gather*}
\item[$(\Leftarrow)$]:
Dokazali bomo, da so vektorji linearno odvisni, če je kakšen linearna kombinacija ostalih (ne nujno prejšnjih). \\

Recimo, da je $v_j=\sum_{k\neq j}\alpha_kv_k$ za nek $\alpha_k\in O$. \\

Potem je $\alpha_1v_1+\ldots+\alpha_{j-1}v_{j-1}+(-1)v_j+\alpha_{j+1}v_{j+1}+\ldots+\alpha_nv_n=0 \Rightarrow$ Vektorji $v_1,\ldots,v_n$ so linearno odvisni.
\end{itemize}
\end{proof}

\begin{definicija}
\textbf{Baza} vektorskega prostora $V$ je množica $B$, ki je hkrati linearno neodvisna in ogrodje.\\ 
\end{definicija}

\newpage 

\noindent PRIMERI:
\begin{itemize}
\item Vsaki trije linearno neodvisni vektorji v $\mathbb{R}^3$ tvorijo bazo
\item Množica $\{e_1,\ldots,e_n\}$ ($e_j=(0,\ldots,0,1,0,\ldots,0$) je baza prostora $O^n$. Pravimo ji \textbf{standardna baza} prostora $O^n$
\item $\{1,x,x^2,\ldots\}$ je (neskončna) baza prostora $\mathbb{R}[X]$.\\
\end{itemize}

\begin{izrek}
Množica $B=\{v_1,\ldots,v_n\}$ je baza vektorskega prostora $V$ natanko takrat, ko vsak $x\in V$ lahko enolično zapišemo v obliki $x=\alpha_1 v_1+\ldots+\alpha_n v_n$, kjer so $\alpha_1,\ldots,\alpha_n \in \mathbb{C}$.\\
\end{izrek}

\begin{proof}~
\begin{itemize}
\item[$(\Rightarrow)$] Naj bo $B$ baza in $x\in V$ poljuben. \\

Ker je $B$ ogrodje, je $x=\alpha_1 v_1+\ldots+\alpha_n v_n$ za neki $\alpha_1,\ldots,\alpha_n \in \mathbb{C}$.\\

\noindent Dokazati je treba še enoličnost tega zapisa.\\

\noindent Recimo, da je $x=\alpha_1 v_1+\ldots + \alpha_n v_n=\beta _1 v_1+\ldots + \beta_n v_n \Rightarrow $ \\

$ \Rightarrow (\alpha_1-\beta_1)v_1+\ldots+(\alpha_n-\beta_n)v_n=0$. \\

\noindent Ker so $v_1,\ldots,v_n$ linearno neodvisni, je $\alpha_1-\beta_1=0,\ldots,\alpha_n-\beta_n=0 \Rightarrow$ \\

$\Rightarrow$ $\alpha_1=\beta_1,\ldots,\alpha_n=\beta_n$

\item[$(\Leftarrow)$] Predpostavimo, da se da vsak vektor $x\in V$ na enoličen način zapisati kot $x=\alpha_1v_1+\ldots+\alpha_nv_n,~\alpha\in O$. \\

\noindent Potem takoj sledi, da je $B$ ogrodje. \\

\noindent Dokažimo še linearno neodvisnost. \\

Naj bo $\alpha_1v_1+\ldots+\alpha_nv_n=0=0\cdot v_1+\ldots+0\cdot v_n$. \\

\noindent Ker je zapis $\alpha_1v_1+\ldots+\alpha_nv_n$ enoličen, je $\alpha_1=0,\alpha_2=0,\ldots,\alpha_n=0$. \\

\noindent Torej so $v_1,\ldots,v_n$ linearno neodvisni.\\
\end{itemize}
\end{proof}

\begin{izrek}
Vsak netrivialen končnorazsežen vektorski prostor ima bazo. Izberemo jo lahko iz poljubnega končnega ogrodja.\\
\end{izrek}

Opomba: Tudi neskončnorazsežni vektorski prostori imajo bazo, kar lahko dokažemo z \linebreak Zornovo lemo.\\

\begin{proof}
Naj bo $\{v_1,\ldots,v_n\}$ poljubno končno ogrodje prostora $V$. \\

Predpostavimo lahko, da je $v_j\neq 0$ za vsak $j$. \\

Zaporedoma od leve proti desni iz ogrodja odstranjujemo vektorje, ki so linearna kombinacija prejšnjih. Na vsakem koraku odstranimo vektor, ki je linearna kombinacija ostalih, zato po odstranitvi še vedno imamo ogrodje. Postopek se po končno korakih konča in dobimo ogrodje $\{u_1,\ldots,u_m\}$. Ker se je postopek končal, noben vektor $u_j$ ni linearna kombinacija prejšnjih. To pa pomeni, da so vektorji $u_1,\ldots,u_m$ linearno neodvisni in torej tvorijo bazo. \\
\end{proof}

\begin{trditev}
Naj bo $\{v_1,\ldots,v_m\}$ poljubno ogrodje vektorskega prostora $V$. Potem nobena \linebreak linearno neodvisna podmnožica prostora $V$ nima več kot $m$ elementov. \\
\end{trditev}

\begin{proof}
Naj bo $\{u_1,\ldots,u_m\}$ linearno neodvisna množica vektorjev v $V$. Dokazu, ki bo \linebreak sledil, rečemo nadomeščanje vektorjev. \\

Ker je $\{v_1,\ldots,v_m\}$ ogrodje, je $u_1\in Lin\{v_1,\ldots,v_m\}$. Zato je $\{u_1,v_1,\ldots,v_m\}$ linearno \linebreak  odvisna množica. \\

Torej je en od vektorjev iz te množice linearna kombinacija prejšnjih. To ni $u_1$, torej je to eden od vektorjev $v_j$. Tega lahko odstranimo in še vedno dobimo ogrodje $\{u_1,v_1+,\ldots,v_{m-1}'\}$. To je ogrodje, zato je $u_2\in Lin\{u_1,v_1+,\ldots,v_{m-1}\}$ in zato je množica $\{u_1,u_2,v_1',\ldots,v_{m-1}\}$ linearno odvisna. \\

Eden od vektorjev iz te množice je linearna kombinacija prejšnjih. To je vektor $v_j'$ za nek $j$, saj sta $u_1$ in $u_2$ linearno neodvisna. Ta $v_j'$ odstranimo in spet dobimo ogrodje $\{u_1,u_2,u_1'',\ldots,u_{m-?}''\}$. To ponavljamo. \\

Recimo, da je $n > m$. \\

Na $m$-tem koraku dobimo ogrodje $\{u_1,\ldots,u_m\}$. \\

Ker je $n>m$ obstaja $u_{??}$ in ker je $\{u_1,\ldots,u_?\}$ ogrodje, je $u_{??}$ linearna kombinacija vektorjev $u_1,\ldots,u_m$. \\

To je v protislovju z linearno neodvisnostjo vektorjev $u_1,\ldots,u_n$. Torej je $n \leq m$. \\
\end{proof}

\begin{posledica}
Vse baze končnorazsežnega vektorskega prostora imajo isto moč. \\
\end{posledica}

\begin{proof}
Naj bosta $B_1$ in $B_2$ bazi prostora $V$ in naj bo $|B_1|=n$ in $|B_2|=m$. \\

$B_1$ je ogrodje, $B_2$ pa linearno neodvisna $\Rightarrow n \geq m$. \\

$B_2$ je tudi ogrodje, $B_1$ pa linearno neodvisna $\Rightarrow m \geq n \Rightarrow m=n$.\\
\end{proof}

\begin{definicija}
Moč baze (končnorazsežnega) vektorskega prostora $V$ se imenuje \textbf{razsežnost} ali \textbf{dimenzija} prostora $V$. Oznaka: dim$V$. \\
\end{definicija}

\begin{trditev}
Vsako linearno neodvisno podmnožico končnorazsežnega vektorskega prostora lahko dopolnimo do baze.\\
\end{trditev}

\begin{proof}
Naj bodo $v_1,\ldots,v_n$ linearno neodvisni vektorji in $B=\{u_1,\ldots,u_m\}$ poljubna baza prostora $V$. \\

Vemo že, da je $m \geq n$. \\

Kot v dokazu prejšnje trditve vektorje $u_i$ zaznamujemo z vektorji $v_j$. \\

Dobimo ogrodje $B'=\{v_1,\ldots,v_n,u_1',\ldots,u_{m-n}'\}$. \\

Iz tega ogrodja lahko izberemo bazo.\\

Ker imajo vse baze $m$ elementov, je $B'$ že baza.\\

 Množico $\{v_1,\ldots,v_n\}$ smo torej dopolnili do baze $B'$. \\
\end{proof}

PRIMER: 
\begin{itemize}
\item Vektorja $(1,1,1)$ in $(0,1,1)$ sta očitno linearno neodvisna. Dopolnimo ju do baze $\mathbb{R}^3$. \\

Izberemo si standardno bazo $B=\{(1,0,0),(0,1,0),(0,0,1)\}$ prostora $\mathbb{R}^3$. Potem je $\{(1,1,1),(0,1,1),(1,0,0),(0,1,0),(0,0,1)\}$ ogrodje prostora $\mathbb{R}^3$. Od leve proti desni odstranjujemo vektorje, ki so linearna kombinacija prejšnjih. Ker sta $(1,1,1)$ in $(0,1,1)$ linearno neodvisna, ju ne odstranimo. \\

$(1,0,0)=(1,1,1)-(0,1,1)\Rightarrow (1,0,0)$ odstranimo. \\

Ali je $(0,1,0)=\alpha(1,1,1)+\beta(0,1,1)$?
\begin{align*}
0 &= \alpha \\
1 &=\alpha + \beta \\
0 & = \alpha+\beta
\end{align*}

To je protislovje $\Rightarrow (1,1,1),(0,1,1)$ in $(0,1,0)$ so linearno neodvisni. \\

$(0,0,1)$ bo zagotovo linearna kombinacija teh \underline{treh linearno neodvisnih} vektorjev v \underline{$\mathbb{R}^3$} $\Rightarrow \{(1,1,1),(0,1,1),(0,1,0)\}$ je baza $\mathbb{R}^3$.\\ 
\end{itemize}

\begin{posledica}~
\begin{enumerate}
\item Če je dim$V=n$ in so vektorji $v_1,\ldots,v_n$ linearno neodvisni, potem tvorijo bazo
\item Naj bo $W$ vektorski podprostor prostora $V$. Potem je dim$W \leq$ dim$V$ in enakost velja le v primeru, ko je $W=V$. \\
\end{enumerate}
\end{posledica}

\begin{proof}~
\begin{enumerate}
\item Ker so $v_1,\ldots,v_n$ linearno neodvisni, jih lahko dopolnimo do baze. \\

Vse baze imajo $n$ elementov, zato ne smemo ničesar dodati.\\

 Torej je $\{v_1,\ldots,v_n\}$ že baza.\\
 
\item V $W$ izberemo bazo $\{w_1,\ldots,w_m\}$. \\

Ta množica je linearno neodvisna, zato jo lahko dopolnimo do baze $\{w_1,\ldots,w_m,v_1,\ldots,v_n\}$ prostora $V$.
\begin{align*}
dimW & = m \\
dimV &=m+n
\end{align*}
Kdaj je dim$W=$dim$V$? $\Leftrightarrow \{w_1,\ldots,w_m\}$ je baza za $V$. \\

$W=\text{Lin}\{w_1,\ldots,w_m\}=V$. 
\end{enumerate}
\end{proof}

\begin{trditev}
Naj bosta $W$ in $U$ vektorska podprostora prostora $V$. Potem je  \linebreak dim$(W+U)=$dim$W+$dim$U-$dim$(W\cap U)$. \\
\end{trditev}

\begin{proof}
Na vajah.\\
\end{proof}

\begin{posledica}
dim$(W \oplus U)=$dim$W+$dim$U$\\
\end{posledica}

\begin{proof}
Če je vsota direktna, je $W\cap U=\{0\}$.\\
\end{proof}

\begin{trditev}
Naj bo $W$ vektorski podprostor končnorazsežnega vektorskega prostora $V$. \linebreak Potem obstaja vektorski podprostor $U$ prostora $V$, da je $W \oplus U=V$. Podprostoru $U$ rečemo \linebreak \textbf{direktni komplement} podprostora $W$. \\
\end{trditev}

\begin{proof}
Če je $W=\{0\}$, vzamemo $U=V$. \\

Če je $W=V$, vzamemo $U=\{0\}$. \\

Predpostavimo, da je $W\neq \{0\}$ in $W\neq V$. \\

Izberemo bazo $\{w_1,\ldots,w_n\}$ prostora $W$ in jo dopolnimo do baze $\{w_1,\ldots,w_n,u_1,\ldots,u_m\}$ prostora $V$.\\

$m>0$, ker je $W\neq V$.\\

Definiramo $U=$Lin$\{u_1,\ldots,u_m\}$. \\

Dokažimo, da je $W\cap U=\{0\}$ in $W+U=V$.\\

Recimo, da je $x\in W \cap U$. Torej je $x=\alpha_1w_1+\ldots+\alpha_nw_n=\beta_1u_1+\ldots+\beta_mu_m$ za neke $\alpha_i,\beta_j\in O\Rightarrow \alpha_1w_1+\ldots+\alpha_nw_n-\beta_1u_1-\ldots\beta_mu_m=0$.\\

Ker je $\{w_1,\ldots,w_n,u_1,\ldots,u_m\}$ baza prostora $V$, je $\alpha_1=\ldots=\alpha_n=\beta_1=\ldots=\beta_n=0\Rightarrow x=0 \Rightarrow W \cap U=\{0\}$. \\

Naj bo $x\in V$ poljuben. Ker je $\{w_1,\ldots,w_n,u_1,\ldots,u_m\}$ baza za V, je  \linebreak 
$x=\alpha_1w_1+\ldots+\alpha_nw_n+\beta_1u_1+\ldots+\beta_mu_m$ za neke $\alpha_i,\beta_j\in O$. \\

Ker je $(\alpha_1w_1+\ldots+\alpha_nw_n)\in W$ in $(\beta_1u_1+\ldots+\beta_mu_m)\in U$, je \linebreak $(\alpha_1w_1+\ldots+\alpha_nw_n+\beta_1u_1+\ldots+\beta_mu_m)\in W+U$.\\
\end{proof}

\newpage 

\begin{trditev}
Naj bosta $U$ in $V$ vektorska prostora nad obsegom $O$ in $A:U\to V$ linearna preslikava. Potem velja:
\begin{itemize}
\item Če je $A$ injektivna, je slika vsake linearno neodvisne množice linearno neodvisna.
\item Če je $A$ surjektivna, je slika vskega ogrodja za $U$ ogrodje za $V$.
\item Če je $A$ bijektivna, je slika vsake baze prostora $U$ baza prostora $V$.\\
\end{itemize}
\end{trditev}

\begin{enumerate}
\item[1)] \begin{proof}
Naj bodo $u_1,\ldots,u_n\in U$ linearno neodvisni in $A:U\to V$ injektivna. \\

Naj bo $\alpha A u_1+\ldots+\alpha_n Au_n=0$ za neke $\alpha_j\in O$. \\

Radi bi dokazali, da je $\alpha_1=\ldots=\alpha_n=0$. \\

Upoštevamo linearnost: $A(\alpha_1u_1+\ldots+\alpha_nu_n)=0$. \\

Ker je $A$ injektivna, je $\alpha_1u_1+\ldots+\alpha_nu_n=0$. \\

Ker so $u_1,\ldots,u_n$ linearno neodvisni, sledi $\alpha_1=\ldots=\alpha_n=0\Rightarrow Au_1,\ldots,Au_n$ so linearno neodvisni.\\
\end{proof}

\item[2)] \begin{proof}
Predpostavimo, da je $\{u_1,\ldots,u_n\}$ ogrodje prostora $U$. \\

Radi bi dokazali, da je $\{Au_1,\ldots,Au_n\}$ ogrodje prostora $V$, če je $A$ surjektivna. \\

Naj bo $x\in V$ poljuben. Ker je $A$ surjektivna, obstaja $y\in U$, da je $x=Ay$. \\

Ker je $\{u_1,\ldots,u_n\}$ ogrodje za $U$, je $y=\alpha_1u_1+\ldots+\alpha_nu_n$ za neke $\alpha_1,\ldots,\alpha_n\in O$ \\

$\Rightarrow x=Ay=A(\alpha_1u_1+\ldots+\alpha_nu_n)=\alpha_1Au_1+\ldots+\alpha_nAu_n$ \\

$\Rightarrow \{Au_1,\ldots,Au_n\}$ je ogrodje za $V$.\\
\end{proof}


\item[3)] \begin{proof} Sledi iz 1) in 2) \end{proof}

\end{enumerate}
\newpage
\begin{izrek} ~

\begin{enumerate}
\item[(1)] Naj bo $V$ $n$-razsežen vektorski prostor nad obsegom $O$ ($n>0$). Potem je $V$ izomorfizem $O^n$
\item[(2)] Končnorazsežna vektorska prostora nad istim obsegom sta izomorfna natanko takrat, ko imata isto razsežnost.\\
\end{enumerate}
\end{izrek}

\begin{proof}~

\begin{enumerate}
\item[(1)] Naj bo $B=\{v_1,\ldots,v_n\}$ baza prostora $V$.\\

Definiramo preslikavo $\phi:O^n\to V$ s predpisom $\phi(\alpha_1,\ldots,\alpha_n)=\alpha_1v_1+\ldots+\alpha_nv_n$. \\

Dokazati moramo, da je $\phi$ linearna in bijektivna. \\

\underline{Linearnost}:
\begin{itemize}
\item Naj bosta $(\alpha_1,\ldots,\alpha_n),(\beta_1,\ldots,\beta_n)\in O^n$ poljubni $n$-terici in naj bosta $\lambda,\mu\in O$. \\

Potem je
\begin{align*}
\phi(\lambda(\alpha_1,\ldots,\alpha_n)+\mu(\beta_1,\ldots,\beta_n)) & = \phi(\lambda \alpha_1+\mu \beta_1,\ldots,\lambda \alpha_n+\mu \beta_n) \\
& = (\lambda \alpha_1+\mu \beta_1)v_1+\ldots+(\lambda \alpha_n+\mu \beta_n)v_n \\
& = \lambda \alpha_1 v_1+\mu \beta_1v_1+\ldots+\lambda \alpha_n v_n+\mu \beta_n v_n \\
& = \lambda (\alpha_1 v_1+\ldots+\alpha_nv_n)+\mu(\beta_1v_1+\ldots+\beta_nv_n) \\
& = \lambda \phi(\alpha_1,\ldots,\alpha_n)+\mu \phi(\beta_1,\ldots,\beta_n)
\end{align*}
\end{itemize}

\underline{Injektivnost}: 
\begin{itemize}
\item Dovolj je dokazati, da je ker$\phi=\{(0,\ldots,0)\}$. \\

Naj bo $(\alpha_1,\ldots,\alpha_n)\in \text{ker}\phi$. \\

Potem je $\alpha_1v_1+\ldots+\alpha_nv_n=0$. \\

Vektorji $v_1,\ldots,v_n$ so linearno neodvisni, zato je $\alpha_1=\ldots=\alpha_n=0$.
\end{itemize}

\underline{Surjektivnost}:
\begin{itemize}
\item Naj bo $x\in V$. \\

Potem je $x=\alpha_1v_1+\ldots+\alpha_nv_n$ za neke $\alpha_1,\ldots,\alpha_n\in O \Rightarrow x=\phi(\alpha_1,\ldots,\alpha_n)\in \text{im}\phi$.\\
\end{itemize}
\newpage
\item[(2)]
\begin{itemize}
\item[$(\Leftarrow)$]
Recimo, da je $n=\text{dim}U=\text{dim}V$.\\

Če je $n=0$, sta $U$ in $V$ triviralna prostora in sta izomorfna. \\

Če je $n>0$, pa je $V$ izomorfen $O^n$, prav tako $U$ (po točki (1)).\\

Izomorfnost je ekvivalenčna relacija, zato sta $U$ in $V$ izomorfna.\\

\item[$(\Rightarrow)$]
Naj bosta $U$ in $V$ izomorfna in $A:U\to V$ izomorfizem. \\

Če je $U=\{0\}$, potem je očitno tudi $V=\{0\}\Rightarrow \text{dim}U=\text{dim}V=0$. \\

Če je $U$ netrivialen, pa ima bazo $B$.\\

Po prejšnji trditvi je $A(B)=$ baza za $V$.\\

Ker je $A$ bijektivna, je $|B|=|A(B)|\Rightarrow \text{dim}U=\text{dim}V$.
\end{itemize}
\end{enumerate}
\end{proof}

\begin{izrek}
Naj bosta $U$ in $V$ končnorazsežna vektorska prostora nad $O$ in $A:U\to V$ linearna preslikava. Potem je $\text{dim}(\text{ker}A)+\text{dim}(\text{im}A)=\text{dim}U$.\\
\end{izrek}

\begin{proof}
Naj bo $\{v_1,\ldots,v_n\}$ baza prostora im$A$.\\

(Predpostavimo, da je slika netrivialna, saj je sicer ker$A=U$ in formula očitno velja).\\

Izberemo poljubne vektorje $u_1,\ldots,u_n\in U$, da je $v_j=Au_j$ za $j=1,\ldots,n$.\\

Izberemo še bazo prostora ker$A:\{w_1,\ldots,w_k\}$.\\

Če dokažemo, da je $B=\{u_1,\ldots,u_n,w_1,\ldots,w_k\}$ baza prostora $U$, bo formula veljala, saj bo dim$(\text{ker}A)=k$, dim$(\text{im}A)=n$ in dim$U=n+k$.\\

\dashuline{B je linearno neodvisna} \\

Recimo, da je $\alpha_1u_1+\ldots+\alpha_nu_n+\beta_1w_1+\ldots+\beta_nw_n=0$.\\

Potem je 
\begin{align*}
0=A0 & =A(\alpha_1u_1+\ldots+\alpha_nu_n+\beta_1w_1+\ldots+\beta_kw_k) \\
& = \alpha_1Au_1+\ldots+\alpha_nAu_n+\beta_1Aw_1+\ldots+\beta_kAw_k \\
& = \alpha_1v_1+\ldots+\alpha_nv_n
\end{align*}

Vektorji $v_1,\ldots,v_n$ so linearno neodvisni, zato je $\alpha_1=\ldots=\alpha_n=0 \Rightarrow \beta_1w_1+\ldots+\beta_kw_k=0$. \\

Vektorji $w_1,\ldots,w_k$ so linearno neodvisni, zato je $\beta_1=\ldots=\beta_n=0$.\\

\dashuline{B je ogrodje} \\

Naj bo $x\in U$ poljuben.\\

Potem je $Ax\in \text{im}A$, zato je $Ax=\alpha_1v_1+\ldots+\alpha_nv_n$. \\

Definirajmo $y=\alpha_1u_1+\ldots+\alpha_nu_n$.\\

Potem je $Ay=\alpha_1v_1+\ldots+\alpha_nv_n=Ax\Rightarrow A(x-y)=0 \Rightarrow x-y\in \text{ker}A$.\\

Ker je $\{w_1,\ldots,w_k\}$ baza za ker$A$, obstajajo $\beta_1,\ldots,\beta_k\in O$, da je $x-y=\beta_1w_1+\ldots+\beta_kw_k\Rightarrow x=\alpha_1u_1+\ldots+\alpha_nu_n+\beta_1w_1+\ldots+\beta_kw_k$.
\end{proof}

\newpage

\section{Kvocientne strukture}

\subsection{Ponovitev relacij}

\textbf{Binarna (dvočlena) relacija} med elementi množice $A$ in $B$ je neprazna podmnožica kartezičnega produkta $A\times B$.\\

Relacijo si lahko mislimo kot posplošitev grafa preslikave $A\to B$. Običajno relacijo razumemo kot zvezo med elementi množice $A$ in $B$. Če je relacija namesto $(x,y)\in R$ pišemo $xRy$. Pravimo, da je \textbf{$x$ v relaciji $R$ z $y$}. \\

Najbolj pogosto primer je, ko je $B=A$. V tem primeru rečemo, da je $R$ \textbf{relacija na $A$}. Pišemo tudi $R\subseteq A\times A$. \\

\subsection{Nekaj lastnosti relacij}

Relacija $R$ na $A$ je \textbf{refleksivna}, kadar velja $xRx$ za vsak $x\in A$. \\

Relacija $R$ na $A$ je \textbf{simetrična}, kadar velja sklep: $xRy\Rightarrow yRx$. \\

Relacija $R$ na $A$ je \textbf{antisimetrična}, kadar velja sklep: $(xRy,yRx)\Rightarrow x=y$.\\

Relacija $R$ na $A$ je \textbf{tranzitivna}, kadar velja sklep: $(xRy,yRz)\Rightarrow xRz$.\\

Relacija $R$ na $A$ je \textbf{relacija delne urejenosti}, kadar je refleksivna, antisimetrična in tranzitivna.\\

Naj $R$ relacija delne urejenosti na $A$. Elementa $x,y\in A$ sta \textbf{primerljiva}, kadar je $xRy$ ali $yRx$. Relacija delne urejenosti, kjer sta vsaka dva elementa primerljiva, se imenuje \textbf{relacija linearna urejenosti}. \\

Naj bo $R$ relacija delne urejenosti na $A$. Element $x\in A$ je \textbf{maksimalen element} glede na relacijo $R$, kadar velja sklep $xRy\Rightarrow x=y$. Element $x\in A$ je \textbf{minimalen element} glede na relacijo $R$, kadar velja sklep $yRx\Rightarrow x=y$. \\

Minimalni in maksimalni elementi ne obstajajo nujno. Če obstajajo, niso nujno enolični. \\

Naj bo $R$ relacija delne urejenosti na $A$. Element $x\in A$ je \textbf{največji element} glede na $R$, kadar velja $yRx$ za vsak $y\in A$. Element $x\in A$ je \textbf{najmanjši element} glede na $R$, kadar velja $xRy$ za vsak $y\in A$. \\

Največji in najmanjši elementi ne obstajajo nujno. Če obstajajo, so enolični. Največji element je vedno maksimalen, najmanjši pa minimalen. Obrat ne velja nujno. \\

\newpage 

PRIMER:
\begin{itemize}
\item $M\neq \emptyset$ naj bo poljubna množica, $|M|\geq 2$. $A$ naj bo množica vseh nepraznih podmnožic množice $M$. $A$ delno uredimo z inkluzijo. \\

Edini maksimalni element je $M$. Je tudi največji, ker so vse podmnožice $M$ (elementi $A$) vsebovani v $M$. \\

Minimalni elementi so množice z 1 elementom. Ni najmanjšega elementa. Če bi bila množica $X$ najmanjši element, bi bila tudi minimalen element, torej množica z 1 elementom: $X=\{a\}$. Ker je $|M|\geq 2,\exists b \in M \backslash \{a\}$. $X=\{a\} \not\subseteq \{b\}$. Torej $X$ ni najmanjši element.\\
\end{itemize} 

\subsection{Ponovitev ekvivalenčne relacije}

Relacija $R$ na $A$ je \textbf{ekvivalnečna relacija}, kadar je refleksivna, simetrična in tranzitivna. Če je $\sim$ ekvivalenčna relacija in je $x\sim y$, je tudi $y\sim x$. V tem primeru pravimo, da sta elementa $x$ in $y$ \textbf{ekvivalentna}. \\

Naj bo $\sim$ ekvivalnečna relacija na množici $A$ in $a\in A$. \textbf{Ekvivalenčni razred} elementa $a$ je množica $[a]_\sim=\{x\in A;x\sim a\}$. Kadar je jasno, za katero relacijo gre, pišemo $[a]$ namesto $[a]_\sim$. Elementu $a$ rečemo \textbf{predstavnik} ekvivalenčnega razreda $[a]$. Vsak element ekvivalenčnega razreda $[a]$ je njegov predstavnik, in vsi predstavniki so med seboj ekvivalentni. \\

\begin{izrek}
Ekvivalenčni razredi razdelijo množico $A$ na unijo paroma disjunktnih ekvivalenčnih razreda, ki so neprazni. Pri tem sta dva elementa množica $A$ ekvivalentna natanko takrat, ko ležita v istem ekvivalenčnem razredu. \\
\end{izrek}

\begin{proof}
LMN \\
\end{proof}

!!!!!!!!!!!!!!!!!!!!Slikala sm shitty in se nč ne vidi!!!!!!!!!!!!!!! \\

\begin{definicija}
Naj bo $\sim$ ekvivalenčna relacija na množici $A$. Množico vseh ekvivalenčnih razredov glede na to relacijo imenujemo \textbf{kvocientna} ali \textbf{faktorska množica} množice $A$ po relaciji $\sim$ in jo označimo $A/_\sim$. Preslikava $q:A\to A/_\sim$, definirana s predpisom $g(a)=[a]$, pa se imenuje \textbf{kanonična kvocientna preslikava}. \\
\end{definicija}

\newpage 

PRIMERI:
\begin{itemize}
\item Na $\mathbb{Z}\times \mathbb{N}$ definiramo relacijo $\sim$ s predpisom $(m,n)\sim(p,q)\Leftrightarrow mq=np$.\\

Refleksivnost: $mn=mn\Rightarrow (m,n)\sim(m,n)$ \\

Simetričnost je tudi očitna.\\

Tranzitivnost: $(m,n)\sim(p,q),(p,q)\sim(r,s)\Rightarrow mq=np,ps=qr$. \\


$mq=np \Rightarrow mqps=npqr \Rightarrow ms=nr, ~\text{če}~ p \neq 0, (m,n)\sim(r,s)$. \\

\dashuline{$ms=nr$}: Če $p=0 \Rightarrow m=0,r=0 \Rightarrow ms=nr \Rightarrow ~ \sim$ je ekvivalenčna relacija:
\begin{align*}
(\mathbb{Z}\times \mathbb{N})/_\sim & = \mathbb{Q} \\
[(m,n)] & \mapsto \frac{m}{n}
\end{align*}

\item Dve usmerjeni daljici v prostoru sta v relaciji $\sim$, kadar sta vzporedni, enako dolgi in kažeta v isto smer. To je ekvivalenčna relacija na množici vseh usmerjenih daljic v prostoru. Kvocientna množica je množica vektorjev. (Spomnimo se natančne definicije vektorja) \\

\item Naj bo $n\in \mathbb{N}$. Na $\mathbb{Z}$ definiramo relacijo $\equiv$ s predpisom $a\equiv b$ ($a$ je kongruentno $b$ po modulu $n$) $\Rightarrow n | a-b$. \\

To je ekvivalenčna relacija.\\

Kvocientna množica je $\mathbb{Z}_n=\mathbb{Z}/_\equiv = \{[0],[1],\ldots,[n-1]\}$. \\

To je množica ostankov pri deljenju z $n$. \\

Namesto $[a]$ v tem primeru pogosto pišemo kar $a$ (če je $0\leq a \leq n-1$), a se moramo zavedati, kaj to pomeni. \\
\end{itemize}

\begin{izrek}
Naj bo $f:A\to B$ preslikava. Na $A$ definiramo relacijo $\sim$ s predpisom $x\sim y \Rightarrow f(x)=f(y)$. Potem velja:
\begin{enumerate}
\item[(1)] $\sim$ je ekvivalenčna relacija na $A$
\item[(2)] Obstaja natanko ena (dobro definirana) preslikava $p:A/_\sim \to B$, da diagram \Smiley{} komutira. Definirana je s predpisom $p([a])=f(a)$. 
\item[(3)] $p$ je injektivna in velja $Z_p=Z_f$. (Dokaz: LMN - kanonični razcep preslikave). Digram: \Smiley{} 
\end{enumerate}
\end{izrek}

\subsection{Usklajenost operacije z ekvivalenčno operacijo}

\begin{definicija}
Na množici $A$ imejmo definirano operacijo $\circ$ in ekvivalenčno relacijo $\sim$. \linebreak Pravimo, da je operacija $\circ$ \textbf{usklajena} z relacijo $\sim$, kadar velja sklep: $(a\sim a',b\sim b')\Rightarrow a\circ b \sim a'\circ b'$. Če je operacija $\circ$ usklajena z relacijo $\sim$, potem lahko na $A/_\sim$ definiramo operacijo $\bullet$ s predpisom $[a]\bullet[b]=[a\circ b]$.\\
\end{definicija}

Ker je operacija na $A/_\sim$ definirana s pomočjo predstavnikov ekvivalenčnih razredov, moramo preveriti dobro definiranost. \\

\dashuline{$[a]=[a'],[b]=[b']\Leftrightarrow [a\circ b]=[a'\circ b']$} \\

$
\begin{cases}
[a]=[a']\Rightarrow a\sim a' \\
[b]=[b']\Rightarrow b\sim b'
\end{cases} $
$ \Rightarrow a\circ b \sim a' \circ b'  \Rightarrow [a\circ b]=[a'\circ b']$ \\

Če operacija $\circ$ ne bi bila usklajena z $\sim$, potem $\bullet$ ne bi bila dobro definirana operacija.\\

PRIMER:
\begin{itemize}
\item Na $\mathbb{Z}\times \mathbb{N}$ imamo definirano relacijo $(m,n)\sim(p,q)\Leftrightarrow mq=np$, seštevanje $(m,n)+(p,g)=(mq+np,nq)$ in množenje $(m,n)\cdot (p,q)=(mp,nq)$. \\

Ali sta $+$ in $\cdot$ usklajena z relacijo $\sim$?\\

Dokažimo, da je to res za seštevanje (sicer glej predavanja iz analize). \\

$(m,n)\sim(m',n'),~(p,g)\sim (p',q')$ \\

$mn'=m'n,~ pq'=p'q$ \\

$(mn'=m'n)/\cdot qq' \rightarrow mn'qq'=m'nqq', \\
(pq'=p'q)/\cdot nn' \rightarrow pq'nn'=p'qnn'$ \\

$mn'qq'+pq'nn'=m'nqq'+p'qnn'$ \\

$(m,n)+(p,q)=(mq+np,nq),~(m',n')+(p',q')=(m'q'+n'p',n'q')$ \\

Ali je $(mq+np,nq)\sim (m'q'+n'p',n'q')$?\\

$\Leftrightarrow mqn'q'+npn'q'=nqm'q'+nqn'p' \rightarrow$ Drži \\

Dokazali smo, da je seštevanje usklajeno z $\sim$. Zato lahko definiramo seštevanje na množici racionalnih števil s predpisom $\frac{m}{n}+\frac{p}{q}=\frac{mn+np}{nq}$, kot smo navajeni. \\

Enako bi dokazali, da je običajno množenje racionalnih števil dobro definirano. \\

\item Na $\mathbb{Z}$ imamo definirani operaciji $+$ in $\cdot$. Naj bo $c\in \mathbb{Z}$ in $a\equiv b \Leftrightarrow n | a-b$. \\

Dokazali smo, da iz $a\equiv a'$ in $b\equiv b'$ sledi $a+b\equiv a'+b'$ in $ab\equiv a'b'$. Zato je na $\mathbb{Z}_n$ dobro definirano seštevanje in množenje s predpisoma $[a]+[b]=[a+b]$ in $[a]\cdot [b]=[a\cdot b]$. 
\end{itemize}

\newpage 

\subsection{Kvocientne grupe Abelovih grup}

$G$ naj bo Abelova grupa in $H$ njena podgrupa. Na $G$ definiramo relacijo $a\sim b \Leftrightarrow a-b\in H$. \\

\begin{trditev}
$\sim$ je ekvivalenčna relacija na $G$.\\
\end{trditev}

\begin{proof}
Refleksivnost: $a\cdot a=0\in H \Rightarrow a \sim q ~ \forall a \in G$ \\

Simetričnost: $a\sim b \Rightarrow a-b\in H \Rightarrow b-a=-(a-b)\in H \Rightarrow b\sim a$ \\

Tranzitivnost: $a\sim b, b\sim c \Rightarrow a-b\in H,b-c\in H$, $a\sim c=(a-b)+(b-c)\in H \Rightarrow a\sim c$ \\

Tudi če $G$ ne bi bila komutativna grupa, bi bila s predpisom $a\sim b \Leftrightarrow ab^{-1} \in H$ definirana ekvivalenčna relacija na $G$. $ab^{-1}\in H \Rightarrow (ab^{-1})^{-1}\in H$ pri čemer je $b=ba^{-1}$.\\
\end{proof}

\begin{izrek}
$G$ naj bo Abelova grupa in $H$ njena podgrupa. Na $G$ definiramo ekvivalečno relacijo $\sim$ s predpisom $a\sim b\Leftrightarrow a-b\in H$. ???????\\
\end{izrek}

PRIMERI:
\begin{itemize}
\item[(1)] Na kvocientni množici $G/_\sim$ lahko definiramo operacijo $+$ s predpisom $[a]+[b]=[a+b]$. 
\item[(2)] Za to operacijo je $G/_\sim$ Abelova grupa. Pravimo ji \textbf{kvocientna} ali \textbf{faktorska grupa} grupe $G$ po podgrupi $H$ in jo označimo $G/_H$. Namesto $[a]$ pišemo $a+H$.
\item[(3)] Kvocientna preslikava \begin{align*} q: G & \Rightarrow G/_H \\ a & \mapsto [a]=a+H \end{align*} je homomorfizem grup. \\
\end{itemize}

\newpage 

\begin{proof} ~
\begin{enumerate}
\item[(1)] Treba je dokazati, da je seštevanje v $G$ usklajeno z relacijo $\sim$. \\

Naj bo $a\sim a'$ in $b\sim b'$. \\

$\Rightarrow a-a'\in H,b-b'\in H \Rightarrow a+b-(a'+b')=a-a'+b-b'\in H$ (ker je $H$ podgrupa). \\

Upoštevali smo, da je $G$ komutativna grupa.\\

$\Rightarrow a+b \sim a'+b'$. \\

Seštevanje v $G$ je usklajeno z $\sim$, zato je seštevanje dobro definirano. 
\item[(2)] Lastnosti operacije se prenesejo iz $G$ na $G/_\sim$. 
\begin{align*}
([a]+[b])+[c] & =[a+b]+[c] \\ 
& = [(a+b)+c] \\
& =[a]+[b+c] \\
& = [a]+([b]+[c]) \\ 
& = [a]+[b] \\
& =[a+b] \\
& = [b+a] \\ 
& =[b]+[a]
\end{align*}

Inverz elementa $[a]$ je $[-a]$.
\item[(3)] $g(a+b)=[a+b]=[a]+[b]=q(a)+q(b)$\\
\end{enumerate}
\end{proof}

Če grupa $G$ ni komutativna, operacija na $G/_\sim=G/_H$ lahko ni dobro definirana. \\

\newpage 

PRIMER: 
\begin{itemize}
\item $G=S_3$, $H=\{id,(1~2)\}$. $H$ je podgrupa v $G$. $a-b \Leftrightarrow ab^{-1}\in H$. \\

$(1~3)\sim(1~2)(1~3)$ \\

$((1~3)((1~2)(1~3))^{-1})=(1~3)(1~3)^{-1}(1~2)^{-1}=(1~2)\in H$\\

$(2~3)\sim(1~2)(2~3)=(1~2~3)$ \\

$(1~3)(2~3)=(1~3~2) \not \sim (1~3~2)(1~2~3)=id$ \\

Operacija ni usklajena z relacijo $\sim$. \\
\end{itemize}

PRIMERI:
\begin{itemize}
\item $H=\{0\}$. $a\sim b \Leftrightarrow a-b\in H \Leftrightarrow a-b=0 \Leftrightarrow a=b$. \\

Ekvivalenčni razredi so enojci $[a]=\{a\}$. \\

$G/_H \cong G$ 

\item $H=G$. Potem so vsi elementi $G$ ekvivalentni $\Rightarrow G/_H$ ima en sam element. Pogosto pišemo $G/_G=\{0\}$. 
\item $G=\mathbb{Z},~n\in \mathbb{N}$, $H=n\cdot \mathbb{Z}=\{nm;m\in \mathbb{Z}\}$. \\

$a\sim b \Leftrightarrow a-b\in H \Leftrightarrow n|a-b \Leftrightarrow a \equiv b$ \\

$\Leftarrow G/_H = G/_\equiv = \mathbb{Z}_n=\{[0],\ldots,[n-1]\}:$ grupa ostankov pri deljenju z $n$. \\

Seštevanje: $[a]+[b]=[a+b]$ \\

V $\mathbb{Z}_5: [3]+[4]=[2]$.\\

Dokazali smo, da je $\mathbb{Z}_n$ celo kolobar za množenje $[a]\cdot [b]=[a\cdot b]$. \\

To NE sledi iz dejstva, da je $G$ kolobar (čeprav je to res) in da je $H$ njegove podkolobar (čeprav je v tem primeru tudi to res).\\ 
\end{itemize}

Če je $K$ kolobar in $H$ njegove podkolobar, $K/_H$ ni nujno kolobar. \\

Primer: 
\begin{itemize}
\item $K=\mathbb{Q},~H=\mathbb{Z},~a\sim b\Leftrightarrow a-b\in \mathbb{Z}$, pri čemer $a\in \mathbb{Q}$ in $b\in \mathbb{Q}$. \\

$\frac{3}{2}-\frac{1}{2}$, $\frac{1}{3} \sim \frac{1}{3}$ \\

$\frac{1}{2}=\frac{3}{2} \cdot \frac{1}{3} \not \sim \frac{1}{2} \cdot \frac{1}{3} = \frac{1}{6}$, saj $\frac{1}{3} - \frac{1}{6} \not \in \mathbb{Z}$ \\

$\Rightarrow$ Množenje na $\mathbb{Q}/_\mathbb{Z}$ ni dobro definirano. \\
\end{itemize}

Spomnimo se: Če je $f:A\to B$ preslikava, je na $A$ s predpisom $a\sim b \Leftrightarrow f(a)=f(b)$ definirana ekvivalenčna relacija in obstaja natanko ena preslikava $p: A/_\sim \to B$, da diagram \Smiley{} komutira. Določena je s predpisom $p([a])=f(a)$. $Z_f=Z_p$, $p$ je injektivna. \\

Oglejmo si primer, ko sta $A$ in $B$ Abelovi grupi in $f:A\to B$ homomorfizem grup. \\

$a\sim b \Leftrightarrow f(a)=f(b) \Leftrightarrow f(a)-f(b) = 0  \Leftrightarrow f(a-b)=0 \Leftrightarrow a-b\in \text{ker}f$. \\

$A/_\sim = A/_\text{ker}f$. \\

\begin{izrek}
Naj bosta $G,H$ Abelovi grupi in $f:G\to H$ homomorfizem grup. Potem velja:
\begin{enumerate}
\item Obstaja natanko en homomorfizem grup $p:G/_\text{ker}f \to H$, da diagram \Smiley{} komutira. Definirana je s predpisom $p([a])=f(a)$. 
\item $p$ je injektiven in velja im$p=$im$f$. V posebnem primeru je $G/_\text{ker}f \cong \text{im}f$. \\

Opomba: Izrek velja tudi, če $G$ in $H$ nista Abelovi (algebra 2).\\ 
\end{enumerate}
\end{izrek}

\begin{proof}
Dokazati je treba le, da je $p$ homomorfizem grup (ostalo že vemo). \\

$p([a]+[b])=p([a+b])=f(a+b)=f(a)+f(b)=p([a])+p([b])$. \\
\end{proof}




\end{document}








\end{document}